\chapter{Serie y transformada de Fourier}

Ambas herramientas permiten descomponer una señal en componentes senoidales, revelando su contenido en frecuencia.
Esta perspectiva frecuencial es especialmente útil cuando se analiza el comportamiento de sistemas LIT.
Permite estudiar cómo cada frecuencia es procesada por un sistema de forma independiente.

Las series de Fourier permiten representar funciones \textbf{periódicas} como una suma infinita de exponenciales complejas.
Los coeficientes de los sumandos van a comprender un espectro frecuencial \textbf{discreto} dado por armónicos.

La transformada de Fourier es una transformación lineal que generaliza este concepto hacia señales \textbf{no periódicas}.
Esta herramienta extiende la idea de las series a un espectro frecuencial \textbf{continuo}, mediante integrales en lugar de series.
La transformada inversa permite reconstruir la señal original.

\section{Condiciones de Dirichlet}

La única condición suficiente y necesaria para que la serie de Fourier converja es
\[
    \fx[x]{t} \in \squaredL
\]
osea, que la energía en un período sea finita:
\[
    \int_0^{T_0} \norm{\fx[x]{t}}^2 \dif t < \infty
\]

Por otro lado, las condiciones de Dirichlet son suficientes, pero no necesarias.
Estas son:

\begin{numset}
    \begin{numitem}{Integrabilidad en el período:}
        $\fx[x]{t}$ debe ser absolutamente integrable en un período:
        \[
            \int_{T_0} \norm{\fx[x]{t}} \, \dif t < \infty
        \]
    \end{numitem}

    \begin{numitem}{Número finito de discontinuidades:}
        $\fx[x]{t}$ puede tener en cada período a lo sumo un número finito de discontinuidades de salto finito.
    \end{numitem}

    \begin{numitem}{Número finito de máximos y mínimos:}
        Dentro de un período, $\fx[x]{t}$ debe tener una cantidad finita de máximos y mínimos.
        Es decir, no puede oscilar infinitamente en un intervalo finito.
    \end{numitem}
\end{numset}

\section{Serie de Fourier}

La \emph{serie de Fourier} es una herramienta matemática que permite desarrollar una función periódica (Def. \ref{defn:funcPeriodCont}) como una suma infinita de senos y cosenos armónicos.
Esta descomposición resulta particularmente útil en el análisis de señales, ya que permite estudiar su comportamiento en el dominio de la frecuencia.

Desde un punto de vista matemático, el desarrollo en series de Fourier se basa en la idea de \emph{proyección ortogonal} en un espacio de funciones.
La familia de exponenciales complejas actúa como una base ortonormal.
En este contexto, los coeficientes de la serie se obtienen mediante el \emph{producto interno} entre la señal y cada una de esas funciones base.
A medida que más términos se suman, mejor se aproxima la señal.
La calidad de la aproximación se puede medir mediante el \emph{error cuadrático medio}.
Al proyectar ortogonalmente, este error se minimiza, lo que garantiza que la serie de Fourier es la mejor aproximación posible.

\begin{mdframed}[style=DefinitionFrame]
    \begin{defn}
        \label{defn:FourierSerieExp}
    \end{defn}
    \cusTi{Serie de Fourier: forma exponencial compleja}
    \[
        \fx[x]{t} = \sum_{\kth=-\infty}^\infty c_\kth \, e^{\iu \, \kth \, \omega_0 \, t}
    \]
    \noTi{donde $c_\kth$ son los coeficientes de Fourier complejos, dados por}
    \[
        c_\kth = \frac{1}{T_0} \int_{T_0} \fx[x]{t} \, e^{-\iu \, \kth \, \omega_0 \, t} \, \dif t
    \]
\end{mdframed}

\begin{mdframed}[style=ExampleFrame]
    \begin{example}
    \end{example}
    Determinar la serie de Fourier y el espectro de
    \[
        \fx[x]{t} = 2 + 4 \fx[\cos]{50 t + \frac{\pi}{2}} + 12 \fx[\cos]{100 t - \frac{\pi}{3}}
    \]

    \concept{Resolución:}

    A partir de la suma de un complejo con su conjugado, podemos expresar $\fx[x]{t}$ en sumas de exponenciales:
    \begin{align*}
        \fx[x]{t}
        & = 2 + \frac{4}{2} \inBrackets{e^{\iu \inParentheses{50 t + \frac{\pi}{2}}} + e^{- \iu \inParentheses{50 t + \frac{\pi}{2}}}} +
        \notag
        \\
        & \hfill \hspace{2cm}
        + \frac{12}{2} \inBrackets{e^{\iu \inParentheses{100 t - \frac{\pi}{3}}} + e^{- \iu \inParentheses{100 t - \frac{\pi}{3}}}}
        \\[2ex]
        & = 2 + 2 \, e^{\iu \inParentheses{50 t + \frac{\pi}{2}}} + 2 \, e^{\iu \inParentheses{- 50 t - \frac{\pi}{2}}} +
        \notag
        \\
        & \hfill \hspace{2.5cm}
        + 6 \, e^{\iu \inParentheses{100 t - \frac{\pi}{3}}} + 6 \, e^{\iu \inParentheses{- 100 t + \frac{\pi}{3}}}
    \end{align*}

    La frecuencia fundamental es $\omega_0 = \radpers[50]$ luego
    \begin{align*}
        \fx[x]{t}
        & = 2 + 2 \, e^{\iu \frac{\pi}{2}} \, e^{\iu \omega_0 t} + 2 \, e^{- \iu \frac{\pi}{2}} \, e^{- \iu \omega_0 t} +
        \notag
        \\
        & \hfill \hspace{2.5cm}
        + 6 \, e^{- \iu \frac{\pi}{3}} \, e^{\iu 2 \, \omega_0 \, t} + 6 \, e^{\iu \frac{\pi}{3}} \, e^{- \iu 2 \, \omega_0 \, t}
        \\[2ex]
        & = 2 \, e^0 + 2 \iu \, e^{\iu \omega_0 t} - 2 \, \iu \, e^{- \iu \omega_0 t} +
        \\
        & \hfill \hspace{0.5cm}
        + (3 - \iu 3 \sqrt{3}) \, e^{\iu 2 \, \omega_0 \, t} + (3 + \iu 3 \sqrt{3}) \, e^{- \iu 2 \, \omega_0 \, t}
    \end{align*}

    Y el espectro es
    \begin{center}
        \includegraphics[width=\linewidth]{./images/ej_serie_fourier.png}
    \end{center}
\end{mdframed}

Reacomodando la sumatoria de la definición \ref{defn:FourierSerieExp} según
\begin{align*}
    \fx[x]{t} &=
    \sum_{\kth=-\infty}^\infty c_\kth \, e^{\iu \, \kth \, \omega_0 \, t}
    \\[1em]
    &= \sum_{\kth=-\infty}^{-1} c_\kth \, e^{\iu \, \kth \, \omega_0 \, t} + c_0 + \sum_{\kth=1}^{\infty} c_\kth \, e^{\iu \, \kth \, \omega_0 \, t}
    \\[1em]
    &= \sum_{\kth=1}^{\infty} c_{-\kth} \, e^{-\iu \, \kth \, \omega_0 \, t} + c_0 + \sum_{\kth=1}^{\infty} c_\kth \, e^{\iu \, \kth \, \omega_0 \, t}
\end{align*}
es posible obtener la siguiente expresión equivalente:
\begin{equation}
    \fx[x]{t} = c_0 + \sum_{\kth=1}^{\infty} \inParentheses{ c_\kth \, e^{\iu \, \kth \, \omega_0 \, t} + c_{-\kth} \, e^{-\iu \, \kth \, \omega_0 \, t} }
    \label{eqn:FourierExpBis}
\end{equation}

Y aplicando la fórmula de Euler:
\begin{align*}
    &
    \scale{0.80}{
    = c_0 + \sum_{\kth=1}^{\infty} \inBraces{c_\kth \inBrackets{\fx[\cos]{\kth \, \omega_0 \, t} + \iu \fx[\sin]{\kth \, \omega_0 \, t}} + c_{-\kth} \inBrackets{\fx[\cos]{\kth \, \omega_0 \, t} - \iu \fx[\sin]{\kth \, \omega_0 \, t}} }
    }
    \\[1em]
    &= c_0 + \sum_{\kth=1}^{\infty} \inBrackets{ \inParentheses{c_\kth + c_{-\kth}} \fx[\cos]{\kth \, \omega_0 \, t} + \iu \inParentheses{c_\kth - c_{-\kth}} \fx[\sin]{\kth \, \omega_0 \, t} }
\end{align*}

\begin{mdframed}[style=DefinitionFrame]
    \begin{defn}
        \label{defn:FourierSerieTrig}
    \end{defn}
    \cusTi{Serie de Fourier: forma trigonométrica}
    \[
        \fx[x]{t} = c_0 + \sum_{\kth=1}^{\infty} \inBrackets{a_\kth \fx[\cos]{\kth \, \omega_0 \, t} + b_\kth \fx[\sin]{\kth \, \omega_0 \, t} }
    \]
    \noTi{donde $a_\kth = c_\kth + c_{-\kth}$ y $b_\kth = \iu \inParentheses{c_\kth - c_{-\kth}}$ son los coeficientes de Fourier trigonométricos.}
\end{mdframed}

Por otro lado, si se aplica la fórmula de Euler pero esta vez en la exponencial de los coeficientes complejos de la definición \ref{defn:FourierSerieExp}:
\begin{align*}
    c_{\pm\kth} &= \frac{1}{T_0} \int_{T_0} \fx[x]{t} \, e^{\mp \iu \, \kth \, \omega_0 \, t} \, \dif t
    \\
    &= \frac{1}{T_0} \int_{T_0} \fx[x]{t} \inBrackets{ \fx[\cos]{\kth \, \omega_0 \, t} \mp \iu \fx[\sin]{\kth \, \omega_0 \, t} } \, \dif t
    \\
    &= \frac{1}{T_0} \int_{T_0} \fx[x]{t} \fx[\cos]{\kth \, \omega_0 \, t} \dif t
    \mp \frac{\iu}{T_0} \int_{T_0} \fx[x]{t} \fx[\sin]{\kth \, \omega_0 \, t} \dif t
\end{align*}

Se definen los términos $\alpha = \int_{T_0} \fx[x]{t} \fx[\cos]{\kth \, \omega_0 \, t} \dif t$ y $\beta = \int_{T_0} \fx[x]{t} \fx[\sin]{\kth \, \omega_0 \, t} \dif t$ para simplificar la notación, tal que:
\[
     c_{\pm\kth} = \frac{\alpha}{T_0} \mp \iu \frac{\beta}{T_0}
\]

Obteniendo las siguientes expresiones para los coeficientes trigonométricos de la definición \ref{defn:FourierSerieTrig}:
\[
    \left\{
    \begin{aligned}
        a_\kth &= c_\kth + c_{-\kth} = \frac{2 \, \alpha}{T_0}
        \\
        b_\kth &= \iu \inParentheses{c_\kth - c_{-\kth}} = \frac{2 \, \beta}{T_0}
    \end{aligned}
    \right.
\]

Esto no es casualidad y viene dado por una propiedad que sale de sumar complejos conjugados.
Podemos comprobarla gráficamente.
Dado que $a_\kth \in \setR$ y $b_\kth \in \setI$ son ortogonales entre sí, los coeficientes $c_{\pm\kth} \in \setC$ no forman cualquier ángulo, sino que $c_\kth$ es el conjugado de $c_{-\kth}$.
A continuación se muestran con flechas en el plano complejo:

\begin{center}
    \def\svgwidth{0.6\linewidth}
    \input{./images/fourier-abc.pdf_tex}
\end{center}

A partir del gráfico anterior, se define el complejo $C_\kth = a_\kth - \iu \, b_\kth$ representado por un punto en el vértice del triángulo gris.

Luego, multiplicando y dividiendo por $\norm{C_\kth}$ la serie de la definición \ref{defn:FourierSerieTrig} se tiene
\[
    \fx[x]{t} = c_0 + \sum_{\kth=1}^{\infty} \norm{C_\kth} \inBrackets{ \dfrac{a_\kth}{\norm{C_\kth}} \fx[\cos]{\kth \, \omega_0 \, t} + \dfrac{b_\kth}{\norm{C_\kth}} \fx[\sin]{\kth \, \omega_0 \, t} }
\]

Por trigonometría se tiene $\fx[\cos]{\theta_\kth} = a_\kth / \norm{C_\kth}$ mientras que $\fx[\sin]{\theta_\kth} = b_\kth / \norm{C_\kth}$ luego:
\[
    \fx[x]{t} = c_0 + \sum_{\kth=1}^{\infty} \norm{C_\kth} \inBrackets{ \fx[\cos]{\theta_\kth} \fx[\cos]{\kth \, \omega_0 \, t} + \fx[\sin]{\theta_\kth} \fx[\sin]{\kth \, \omega_0 \, t} }
\]

Aplicando $\fx[\cos]{\alpha \pm \beta} = \fx[\cos]{\alpha} \fx[\cos]{\beta} \mp \fx[\sin]{\alpha} \fx[\sin]{\beta}$ se infiere:

\begin{mdframed}[style=DefinitionFrame]
    \begin{defn}
        \label{defn:FourierSerieArm}
    \end{defn}
    \cusTi{Serie de Fourier: forma armónica}
    \[
        \fx[x]{t} = c_0 + \sum_{\kth=1}^{\infty} \norm{C_\kth} \fx[\cos]{\kth \, \omega_0 \, t - \theta_\kth}
    \]
    \noTi{donde $C_\kth$ son los coeficientes de Fourier expresados en forma polar, tal que:}
    \[
    \left\{
    \begin{aligned}
        \norm{C_\kth} = \sqrt{a_\kth^2 + b_\kth^2}
        \\
        \theta_\kth = \fx[\artan]{\frac{b_\kth}{a_\kth}}
    \end{aligned}
    \right.
\]
\end{mdframed}

\section{Coeficientes de Fourier típicos}

\subsection{Tren de pulsos rectangulares}

Para un pulso rectangular genérico
\[
    \fx[x]{t} = A \, \fx[\Pi]{\frac{t-t_0}{\tau}}
\]
los coeficientes están dados, según la definición \ref{defn:FourierSerieExp}, como sigue:
\begin{align*}
    c_\kth &= \frac{1}{T_0} \int_{T_0} \fx[x]{t} \, e^{-\iu \, \kth \, \omega_0 \, t} \, \dif t
    \\[1ex]
    &= \frac{A}{T_0} \int_{-\frac{\tau}{2}}^{\frac{\tau}{2}} e^{-\iu \, \kth \, \omega_0 \, t} \, \dif t
    \\[1ex]
    &= \frac{A}{T_0} \barrow{\dfrac{e^{-\iu \, \kth \, \omega_0 \, t}}{-\iu \, \kth \, \omega_0}}{-\frac{\tau}{2}}{\frac{\tau}{2}}
    \\[1ex]
    &= \frac{A}{-2 \, \pi \, \iu \, \kth } \inParentheses{ e^{ -\iu \, \omega_0 \, \kth \frac{\tau}{2}} - e^{ \iu \, \omega_0 \, \kth \frac{\tau}{2}} }
    \\[1ex]
    &= \frac{A}{\pi \, \kth} \, \fx[\sin]{\frac{\omega_0 \, \kth \, \tau}{2}}
    \\[1ex]
    &= \frac{A}{\pi \, \kth} \, \fx[\sin]{\frac{\pi \, \kth \, \tau}{T_0}}
    \\[1ex]
    &= \frac{A \, \tau}{T_0} \cdot
    \frac{ \fx[\sin]{\dfrac{\pi \, \kth \, \tau}{T_0}} }{ \dfrac{\pi \, \kth \, \tau}{T_0} }
\end{align*}
luego
\begin{equation}
    c_\kth = \frac{A \, \tau}{T_0} \, \fx[\sinc]{\frac{\pi \, \kth \, \tau}{T_0}}
    \label{eqn:ckPulso}
\end{equation}

\begin{mdframed}[style=ExampleFrame]
    \begin{example}
    \end{example}
    \cusTi{De la proyección ortogonal a la serie}
    \begin{formatI}
        Calcular la proyección ortogonal de una onda cuadrada.
        Obtener los coeficientes de Fourier complejos.
        Demostrar que son equivalentes a una serie deducida a partir de la proyección.
    \end{formatI}
    \vspace{1em}
    Sea $x$ una función por partes dada por
    \[
        \fx[x]{t} =
        \left\{
        \begin{matrix}
            0 & \text{si } -\pi < t < -\frac{\pi}{2}
            \\
            1 & \text{si } -\frac{\pi}{2} < t < \frac{\pi}{2}
            \\
            0 & \text{si } \frac{\pi}{2} < t < \pi
        \end{matrix}
        \right.
    \]
    
    Sea $B_S$ una base de polinomios trigonométricos.
    A saber:
    \[
        B_S = \inBraces{1; \fx[\sin]{t}; \fx[\cos]{t}; \fx[\sin]{2t}; \dots; \fx[\sin]{\kth t}; \fx[\cos]{\kth t}}
    \]
    
    De manera que la proyección de $x$ sobre $S$ es:
    \begin{multline*}
        \fx[\proy_S]{x} =
        \\
        = \frac{1}{2} + \frac{2}{\pi} \inBrackets{\fx[\cos]{t} - \dfrac{\fx[\cos]{3t}}{3} + \dfrac{\fx[\cos]{5t}}{5} - \dfrac{\fx[\cos]{7t}}{7} \cdots}
    \end{multline*}

    Y puede ser expresada por la siguiente serie:
    \begin{equation}
        \fx[\proy_S]{x} = \frac{1}{2} + \frac{2}{\pi} \sum_{\kth=1}^\infty \frac{\inParentheses{-1}^{\kth+1}}{2 \, \kth - 1} \, \fx[\cos]{\inParentheses{2 \, \kth -1} t}
        \label{eqn:FourierProy}
    \end{equation}

    Reemplazando $A=1$, $T_0=2\,\pi$, $\tau=\pi$ en la ecuación \ref{eqn:ckPulso} obtenida para los $c_\kth$ de un pulso, queda
    \[
        c_\kth = \frac{ \fx[\sin]{\frac{\kth \, \pi}{2}} }{\kth \, \pi}
    \]

    Según la ecuación \ref{eqn:FourierExpBis} se tiene:
    \begin{align*}
        \fx[x]{t} &= c_0 + \sum_{\kth=1}^{\infty} \inParentheses{ c_\kth \, e^{\iu \, \kth \, \omega_0 \, t} + c_{-\kth} \, e^{-\iu \, \kth \, \omega_0 \, t} }
        \\
        &= c_0 + \sum_{\kth=1}^{\infty} \frac{\fx[\sin]{\frac{\kth \, \pi}{2}}}{\kth \, \pi} \inParentheses{ e^{\iu \, \kth \, \omega_0 \, t} + e^{-\iu \, \kth \, \omega_0 \, t} }
    \end{align*}

    Obteniendo el desarrollo en series de Fourier para una onda cuadrada:
    \[
        \fx[x]{t} = c_0 + 2 \sum_{\kth=1}^{\infty} \frac{\fx[\sin]{\frac{\kth \, \pi}{2}}}{\kth \, \pi} \, \fx[\cos]{\kth \, \omega_0 \, t}
    \]

    Observar que este desarrollo es equivalente a la ecuación \ref{eqn:FourierProy}, ya que $\fx[\sin]{\frac{\kth \, \pi}{2}}$ vale 0 para los $\kth$ pares y $\pm1$ para los $\kth$ impares.
\end{mdframed}

\section{Propiedades de la serie de Fourier}

\begin{mdframed}[style=PropertyFrame]
    \begin{prop}
    \end{prop}
    Los coeficientes $c_\kth^\nth$ de la serie de Fourier de la n-ésima derivada están dados por
    \[
        c_\kth^\nth = c_\kth \cdot \inParentheses{\frac{\iu \, 2 \, \pi \, \kth}{T_0}}^\nth
    \]
\end{mdframed}

\section{La transformada y su inversa}

La transformada de Fourier extiende el concepto de serie de Fourier a funciones no periódicas.
Permite descomponer señales en una superposición \textbf{continua} de exponenciales complejas.

Se tiene una señal $\fx[x]{t}$ arbitraria, de duración finita, extendible por cero fuera de su dominio, pero no periódica:

\begin{center}
    \def\svgwidth{0.8\linewidth}
    \input{./images/transformada_fourier_1.pdf_tex}
\end{center}

Como $\fx[x]{t}$ no es periódica, no tiene desarrollo en series de Fourier.
Pero podemos definir $\fx[\tilde{x}]{t}$ como la extensión periódica de $\fx[x]{t}$:

\begin{center}
    \def\svgwidth{0.8\linewidth}
    \input{./images/transformada_fourier_2.pdf_tex}
\end{center}

La serie de Fourier de $\fx[\tilde{x}]{t}$ converge, por tener período $T_0$ y ser $\squaredL$.
Al calcularla, considerar en la definición \ref{defn:FourierSerieExp} el cambio de variable
\[
    \omega = \kth \, \omega_0
\]
tanto para la serie, que quedaría dada por
\begin{equation}
    \fx[\tilde{x}]{t}
    = \sum_{\kth=-\infty}^\infty
    c_\kth \, \complexExp
    \label{eqn:serieFourierOmega}
\end{equation}
como para la expresión de los coeficientes:
\begin{equation}
    \tilde{c_\kth} =
    \frac{1}{T_0}
    \int_{-\frac{T_0}{2}}^{\frac{T_0}{2}} \fx[x]{t} \, e^{- \iu \omega t}
    \, \dif t
    \label{eqn:coefFourierOmega}
\end{equation}

Reemplazando la ecuación \ref{eqn:coefFourierOmega} en la ecuación \ref{eqn:serieFourierOmega}, queda
\[
    \fx[\tilde{x}]{t}
    = \sum_{\kth=-\infty}^\infty
    \inBrackets{
    \frac{1}{T_0}
    \int_{-\frac{T_0}{2}}^{\frac{T_0}{2}} \fx[x]{t} \, e^{- \iu \omega t}
    \, \dif t
    }
    \, \complexExp
\]

Y dado que
\[
    \fx[x]{t} = \lim_{T_0 \to \infty} \fx[\tilde{x}]{t}
\]
inferimos que $\fx[x]{t}$ va a estar dada por:
\begin{equation}
    \fx[x]{t}
    = \lim_{\omega_0 \to 0}
    \sum_{\kth=-\infty}^\infty
    \inBrackets{ \frac{\omega_0}{2 \pi} \int_{-\infty}^{\infty} \fx[x]{\tau} \, e^{- \iu \omega \tau} \, \dif \tau }
    \, \complexExp
    \label{eqn:serieFourierNoPeriodica}
\end{equation}

En la serie de $\fx[\tilde{x}]{t}$, $\omega = k \, \omega_0$ solo puede generar sumandos armónicos (múltiplos de $\omega_0$).
Pero en la serie de $\fx[x]{t}$, $\omega$ puede tomar cualquier valor real, dado que $\omega_0 \to 0$.
Por lo tanto, los coeficientes de $\fx[x]{t}$ son continuos y están dados por
\[
    c_\kth =
    \frac{\omega_0}{2 \pi}
    \int_{-\infty}^\infty \fx[x]{t} \, e^{- \iu \omega t}
    \, \dif t
    \quad \text{con } \omega_0 \to 0
\]

Quedando definido el \emph{espectro frecuencial} de $\fx[x]{t}$ como el límite de los coeficientes de $\fx[\tilde{x}]{t}$ escalados:

\begin{mdframed}[style=DefinitionFrame]
    \begin{defn}
        \label{defn:espectroFourier}
    \end{defn}
    \cusTi{Espectro frecuencial}

    El espectro de frecuencias de $\fx[x]{t}$ es
    \[
        \fx[X]{\omega} = \lim_{T_0 \to \infty} \tilde{c_k} \, T_0
    \]
    donde $\tilde{c_k}$ son los coeficientes de Fourier de la extensión periódica de $\fx[x]{t}$.
\end{mdframed}

Esta definición es una motivación heurística que conecta la serie de Fourier con la transformada de Fourier, cuya definición formal es:

\begin{mdframed}[style=DefinitionFrame]
    \begin{defn}
        \label{defn:FourierTrans}
    \end{defn}
    \cusTi{Transformada de Fourier}
    \[
        \fourier{\fx[x]{t}}
        = \int_{-\infty}^\infty \fx[x]{t} \, e^{- \iu \omega t} \, \dif t
        = \fx[X]{\omega}
    \]
\end{mdframed}

En la ecuación \ref{eqn:serieFourierNoPeriodica}, la suma sobre frecuencias discretas $\omega = k \, \omega_0$ se aproxima a una integral sobre el dominio continuo $\omega \in \setR$.
Y $\omega_0$ juega el papel de $\dif \omega$ en el límite de Riemann, que da origen a la transformada inversa.

\begin{mdframed}[style=DefinitionFrame]
    \begin{defn}
        \label{defn:FourierTransInv}
    \end{defn}
    \cusTi{Transformada de Fourier inversa}
    \[
        \ffourier{\fx[X]{\omega}}
        = \frac{1}{2 \, \pi} \int_{-\infty}^\infty \fx[X]{\omega} \, \complexExp \, \dif \omega
        = \fx[x]{t}
    \]
\end{mdframed}

Tanto la transformada como su inversa se pueden expresar en términos de $f$ en vez de $\omega$, según la relación
\[
    \begin{cases}
        \omega &= 2 \, \pi \, f
        \\
        \dif \omega &= 2 \, \pi \, \dif f
    \end{cases}
\]

\section{Propiedades de la transformada}

\subsection{Superposición}

Debido a que la transformada de Fourier es una transformación lineal, se verifica que:

\begin{mdframed}[style=PropertyFrame]
    \begin{prop}
    \end{prop}
    Dada la señal
    \[
        \fx[x]{t} = a_1 \, \fx[x_1]{t} + a_2 \, \fx[x_2]{t} + \dots + a_\nth \, \fx[x_\nth]{t}
    \]
    con $a_1, a_2 \dots a_\nth$ constantes, su transformada de Fourier estará dada por
    \[
        \fx[X]{\omega} = a_1 \, \fx[X_1]{\omega} + a_2 \, \fx[X_2]{\omega} + \dots + a_\nth \, \fx[X_\nth]{\omega}
    \]
\end{mdframed}

\subsection{Desplazamiento en tiempo}

La transformada de una función desplazada en tiempo está dada, por definición, por
\[
    \fourier{\fx[x]{t - t_0}}
    = \int_{-\infty}^\infty \fx[x]{t - t_0} \, e^{- \iu \omega t} \, \dif t
\]

Haciendo el cambio de variable
\[
    \begin{cases}
        u = t - t_0 & \implies t = u + t_0
        \\
        \deriv[u]{t} = 1 & \implies \dif t = \dif u
    \end{cases}
\]
se tiene
\begin{align*}
    \fourier{\fx[x]{t - t_0}}
    & = \int_{-\infty}^{\infty} \fx[x]{u} \, e^{- \iu \omega (u + t_0)} \dif u
    \\[1ex]
    & = \int_{-\infty}^{\infty} \fx[x]{u} \, e^{- \iu \omega u} \, e^{- \iu \omega t_0} \dif u
    \\[1ex]
    & = e^{- \iu \omega t_0} \, \int_{-\infty}^{\infty} \fx[x]{u} \, e^{- \iu \omega u} \dif u
\end{align*}

Obteniendo así:

\begin{mdframed}[style=PropertyFrame]
    \begin{prop}
    \end{prop}
    \[
        \fourier{\fx[x]{t - t_0}} = \fourier{\fx[x]{t}} \, e^{- \iu \omega t_0}
    \]
\end{mdframed}

\begin{mdframed}[style=ExampleFrame]
    \begin{example}
    \end{example}
    Obtener la transformada de Fourier de
    \[
        \fx[x]{t} = A \, \pulse[- \frac{t_0}{2}]{t}{t_0} - A \, \pulse[\frac{t_0}{2}]{t}{t_0} % TODO: Redefinir comando \pulse
    \]

    \concept{Resolución:}

    Sea
    \[
        \fx[x]{t} = \fx[x_1]{t} + \fx[x_2]{t}
    \]
    tal que
    \begin{gather*}
        \fx[x_1]{t} = A \, \pulse[- \frac{t_0}{2}]{t}{t_0} % TODO: Redefinir comando \pulse
        \\[1ex]
        \fx[x_2]{t} = - A \, \pulse[\frac{t_0}{2}]{t}{t_0}
    \end{gather*}

    La transformada del pulso de la izquierda es
    \[
        \fx[X_1]{f} = A \, t_0 \, \fx[\sinc]{\pi f t_0} \, e^{\iu \omega \frac{t_0}{2}}
    \]
    y la transformada del pulso de la derecha es
    \[
        \fx[X_2]{f} = - A \, t_0 \, \fx[\sinc]{\pi f t_0} \, e^{- \iu \omega \frac{t_0}{2}}
    \]
    de manera que la transformada de $\fx[x]{t}$ es
    \begin{align*}
        \fx[X]{f}
        & = \fx[X_1]{f} + \fx[X_2]{f}
        \\[1ex]
        & = A \, t_0 \, \fx[\sinc]{\pi f t_0} \inParentheses{e^{\iu \omega \frac{t_0}{2}} - e^{- \iu \omega \frac{t_0}{2}}}
        \\[1ex]
        & = A \, t_0 \, \fx[\sinc]{\pi f t_0} \inParentheses{e^{\iu \pi f t_0} - e^{- \iu \pi f t_0}}
        \\[1ex]
        & = 2 \iu \, A \, t_0 \, \fx[\sinc]{\pi f t_0} \, \fx[\sin]{\pi f t_0}
        \\[1ex]
        & = 2 \iu \, A \, \pi \, f \, t_0^2 \, \fx[\sinc^2]{\pi f t_0}
        \\[1ex]
        & = \iu \, A \, \omega \, t_0^2 \, \fx[\sinc^2]{\frac{\omega \, t_0}{2}}
    \end{align*}
\end{mdframed}

\subsection{Teorema de modulación}

La transformada inversa de una función desplazada en frecuencia está dada, por definición, por
\[
    \ffourier{\fx[X]{\omega - \omega_c}}
    = \frac{1}{2 \, \pi} \int_{-\infty}^\infty \fx[X]{\omega - \omega_c} \, \complexExp \, \dif \omega
\]

Haciendo el cambio de variable
\[
    \begin{cases}
        u = \omega - \omega_c & \implies \omega = u + \omega_c
        \\
        \deriv[u]{\omega} = 1 & \implies \dif \omega = \dif u
    \end{cases}
\]
se tiene
\begin{align*}
    \ffourier{\fx[X]{\omega - \omega_c}}
    & = \frac{1}{2 \, \pi} \int_{-\infty}^\infty \fx[X]{u} \, e^{\iu (u + \omega_c) t} \, \dif u
    \\[1ex]
    & = \frac{1}{2 \, \pi} \int_{-\infty}^\infty \fx[X]{u} \, e^{\iu u t} \, e^{\iu \omega_c t} \, \dif u
    \\[1ex]
    & =  e^{\iu \omega_c t} \, \frac{1}{2 \, \pi} \int_{-\infty}^\infty \fx[X]{u} \, e^{\iu u t} \, \dif u
\end{align*}

Obteniendo así:

\begin{mdframed}[style=PropertyFrame]
    \begin{prop}
    \end{prop}
    \[
        \ffourier{\fx[X]{\omega - \omega_c}} = \ffourier{\fx[X]{\omega}} \, e^{\iu \omega_c t}
    \]
\end{mdframed}

Que es equivalente a la siguiente expresión, obtenida al aplicar la transformada a ambos miembros de la igualdad:
\[
    \fx[X]{\omega - \omega_c} = \fourier{\fx[x]{t} \, e^{\iu \omega_c t}}
\]

\begin{mdframed}[style=ExampleFrame]
    \begin{example}
    \end{example}
    Obtener la transformada de Fourier de
    \[
        \fx[x]{t} = 10 \, \pulse[70]{t}{140} \, \fx[\cos]{\frac{\pi}{10} t}
    \]

    \concept{Resolución:}

    Primero vamos a reescribir la señal en términos de exponenciales complejas:
    \begin{align*}
        \fx[x]{t} &= 5 \, \pulse[70]{t}{140} \, \inParentheses{e^{\iu \frac{\pi}{10} t} + e^{- \iu \frac{\pi}{10} t}}
        \\
        \fx[x]{t} &= \fx[p]{t} \, e^{\iu \frac{\pi}{10} t} + \fx[p]{t} \, e^{- \iu \frac{\pi}{10} t}
    \end{align*}

    donde
    \[
        \fx[p]{t} = 5 \, \pulse[70]{t}{140}
    \]

    Por superposición, la transformada de $\fx[x]{t}$ va a estar dada por
    \[
        \fx[X]{\omega} = \fx[X_1]{\omega} + \fx[X_2]{\omega}
    \]
    donde
    \[
        \begin{cases}
            \fx[X_1]{\omega} = \fourier{\fx[p]{t} \, e^{\iu \frac{\pi}{10} t}}
            \\
            \fx[X_2]{\omega} = \fourier{\fx[p]{t} \, e^{- \iu \frac{\pi}{10} t}}
        \end{cases}
    \]

    Notar, según el teorema de modulación, que
    \[
        \begin{cases}
            \fx[P]{\omega - \frac{\pi}{10}} = \fourier{\fx[p]{t} \, e^{\iu \frac{\pi}{10} t}}
            \\
            \fx[P]{\omega + \frac{\pi}{10}} = \fourier{\fx[p]{t} \, e^{- \iu \frac{\pi}{10} t}}
        \end{cases}
    \]
    luego
    \[
        \begin{cases}
            \fx[X_1]{\omega} = \fx[P]{\omega - \frac{\pi}{10}}
            \\
            \fx[X_2]{\omega} = \fx[P]{\omega + \frac{\pi}{10}}
        \end{cases}
    \]
    donde $\fx[P]{\omega} = \fourier{\fx[p]{t}}$ está dado por:
    \[
        \fx[P]{\omega} = 700 \fx[\sinc]{70 \, \omega} \, e^{- \iu 70 \, \omega}
    \]

    Finalmente, $\fourier{\fx[x]{t}}$ es
    \[
        \fx[X]{\omega}
        = \fx[P]{\omega - \frac{\pi}{10}} + \fx[P]{\omega + \frac{\pi}{10}}
    \]
    \begin{multline*}
        \fx[X]{\omega}
        = 700 \fx[\sinc]{70 \, \omega - 7 \pi} \, e^{- \iu (70 \, \omega - 7 \pi)} +
        \\
        + 700 \fx[\sinc]{70 \, \omega + 7 \pi} \, e^{- \iu (70 \, \omega + 7 \pi)}
    \end{multline*}
\end{mdframed}

\subsection{Dualidad}

La transformada inversa con $t \longrightarrow -t$ es
\[
    \fx[x]{-t}
    = \frac{1}{2 \, \pi}
    \int_{-\infty}^\infty \fx[X]{\omega} \, e^{- \iu \omega t}
    \, \dif \omega
\]

Cambiando $t \longrightarrow \omega$ y $\omega \longrightarrow t$ simultáneamente queda
\[
    \fx[x]{-\omega}
    = \frac{1}{2 \, \pi}
    \int_{-\infty}^\infty \fx[X]{t} \, e^{- \iu \omega t}
    \, \dif t
\]
donde
\[
    \int_{-\infty}^\infty \fx[X]{t} \, e^{- \iu \omega t}
    \, \dif t
    = \fourier{\fx[X]{t}}
\]

\begin{mdframed}[style=PropertyFrame]
    \begin{prop}
    \end{prop}
    \[
        \fourier{\fx[x]{t}} = \fx[X]{\omega}
        \iff
        \fourier{\fx[X]{t}} = 2 \, \pi \, \fx[x]{-\omega}
    \]
\end{mdframed}

\section{Transformadas de Fourier típicas}

\subsection{Pulso rectangular}

En la ecuación \ref{eqn:ckPulso} se obtuvieron los coeficientes de un tren de pulsos de ancho $\tau$ y altura $A$:
\[
    \tilde{c_\kth} = \frac{A \, \tau}{T_0} \, \fx[\sinc]{\pi \, \kth \, f_0 \, \tau}
\]

Según la definición \ref{defn:espectroFourier}, podemos obtener la transformada de un solo pulso a partir del límite de los coeficientes escalados por $T_0$.
Esto es:
\[
    \fx[X]{f} = A \, \tau \, \fx[\sinc]{\pi f \tau}
\]
o bien, en radianes:
\[
    \fx[X]{\omega} = A \, \tau \, \fx[\sinc]{\frac{\omega \tau}{2}}
\]

Observar que $X \in \setR$ para todas las frecuencias.
Por este motivo, podríamos representarlo con un solo gráfico:

% TODO: Gráfico de X(f). Requiere desarrollar funcionalidad en dsp-package.

O bien graficar su espectro bilateral:

% TODO: Gráficos de |X(f)| y fase(X(f)). Requiere desarrollar funcionalidad en dsp-package.