\chapter{Transformada de Fourier}

\section{Serie de Fourier}

Las \emph{condiciones de Dirichlet} son suficientes (pero no necesarias) para que una función $\fx[x]{t}$ de período $T_0$ tenga desarrollo en series de Fourier.

\begin{numset}
    \begin{numitem}{Integrabilidad en el período:}
        $\fx[x]{t}$ debe ser absolutamente integrable en un período:
        \[
            \int_{T_0} \norm{\fx[x]{t}} \, \dif t < \infty
        \]
    \end{numitem}

    \begin{numitem}{Número finito de discontinuidades:}
        $\fx[x]{t}$ puede tener en cada período a lo sumo un número finito de discontinuidades de salto finito.
    \end{numitem}

    \begin{numitem}{Número finito de máximos y mínimos:}
        Dentro de un período, $\fx[x]{t}$ debe tener una cantidad finita de máximos y mínimos.
        Es decir, no puede oscilar infinitamente en un intervalo finito.
    \end{numitem}
\end{numset}

\begin{mdframed}[style=DefinitionFrame]
    \begin{defn}
        \label{defn:FourierSerieExp}
    \end{defn}
    \cusTi{Serie de Fourier: forma exponencial compleja}
    \[
        \fx[x]{t} = \sum_{\kth=-\infty}^\infty c_\kth \, e^{\iu \, \kth \, \omega_0 \, t}
    \]
    \noTi{donde $c_\kth$ son los coeficientes de Fourier complejos, dados por}
    \[
        c_\kth = \frac{1}{T_0} \int_{T_0} \fx[x]{t} \, e^{-\iu \, \kth \, \omega_0 \, t} \, \dif t
    \]
\end{mdframed}

Reacomodando la sumatoria de la definición \ref{defn:FourierSerieExp} según
\begin{align*}
    \fx[x]{t} &=
    \sum_{\kth=-\infty}^\infty c_\kth \, e^{\iu \, \kth \, \omega_0 \, t}
    \\[1em]
    &= \sum_{\kth=-\infty}^{-1} c_\kth \, e^{\iu \, \kth \, \omega_0 \, t} + c_0 + \sum_{\kth=1}^{\infty} c_\kth \, e^{\iu \, \kth \, \omega_0 \, t}
    \\[1em]
    &= \sum_{\kth=1}^{\infty} c_{-\kth} \, e^{-\iu \, \kth \, \omega_0 \, t} + c_0 + \sum_{\kth=1}^{\infty} c_\kth \, e^{\iu \, \kth \, \omega_0 \, t}
\end{align*}
es posible obtener la siguiente expresión equivalente:
\begin{equation}
    \fx[x]{t} = c_0 + \sum_{\kth=1}^{\infty} \inParentheses{ c_\kth \, e^{\iu \, \kth \, \omega_0 \, t} + c_{-\kth} \, e^{-\iu \, \kth \, \omega_0 \, t} }
    \label{eqn:FourierExpBis}
\end{equation}

Y aplicando la fórmula de Euler:
\begin{align*}
    &
    \scale{0.80}{
    = c_0 + \sum_{\kth=1}^{\infty} \inBraces{c_\kth \inBrackets{\fx[\cos]{\kth \, \omega_0 \, t} + \iu \fx[\sin]{\kth \, \omega_0 \, t}} + c_{-\kth} \inBrackets{\fx[\cos]{\kth \, \omega_0 \, t} - \iu \fx[\sin]{\kth \, \omega_0 \, t}} }
    }
    \\[1em]
    &= c_0 + \sum_{\kth=1}^{\infty} \inBrackets{ \inParentheses{c_\kth + c_{-\kth}} \fx[\cos]{\kth \, \omega_0 \, t} + \iu \inParentheses{c_\kth - c_{-\kth}} \fx[\sin]{\kth \, \omega_0 \, t} }
\end{align*}

\begin{mdframed}[style=DefinitionFrame]
    \begin{defn}
        \label{defn:FourierSerieTrig}
    \end{defn}
    \cusTi{Serie de Fourier: forma trigonométrica}
    \[
        \fx[x]{t} = c_0 + \sum_{\kth=1}^{\infty} \inBrackets{a_\kth \fx[\cos]{\kth \, \omega_0 \, t} + b_\kth \fx[\sin]{\kth \, \omega_0 \, t} }
    \]
    \noTi{donde $a_\kth = c_\kth + c_{-\kth}$ y $b_\kth = \iu \inParentheses{c_\kth - c_{-\kth}}$ son los coeficientes de Fourier trigonométricos.}
\end{mdframed}

Por otro lado, si se aplica la fórmula de Euler pero esta vez en la exponencial de los coeficientes complejos de la definición \ref{defn:FourierSerieExp}:
\begin{align*}
    c_{\pm\kth} &= \frac{1}{T_0} \int_{T_0} \fx[x]{t} \, e^{\mp \iu \, \kth \, \omega_0 \, t} \, \dif t
    \\
    &= \frac{1}{T_0} \int_{T_0} \fx[x]{t} \inBrackets{ \fx[\cos]{\kth \, \omega_0 \, t} \mp \iu \fx[\sin]{\kth \, \omega_0 \, t} } \, \dif t
    \\
    &= \frac{1}{T_0} \int_{T_0} \fx[x]{t} \fx[\cos]{\kth \, \omega_0 \, t} \dif t
    \mp \frac{\iu}{T_0} \int_{T_0} \fx[x]{t} \fx[\sin]{\kth \, \omega_0 \, t} \dif t
\end{align*}

Se definen los términos $\alpha = \int_{T_0} \fx[x]{t} \fx[\cos]{\kth \, \omega_0 \, t} \dif t$ y $\beta = \int_{T_0} \fx[x]{t} \fx[\sin]{\kth \, \omega_0 \, t} \dif t$ para simplificar la notación, tal que:
\[
     c_{\pm\kth} = \frac{\alpha}{T_0} \mp \iu \frac{\beta}{T_0}
\]

Obteniendo las siguientes expresiones para los coeficientes trigonométricos de la definición \ref{defn:FourierSerieTrig}:
\[
    \left\{
    \begin{aligned}
        a_\kth &= c_\kth + c_{-\kth} = \frac{2 \, \alpha}{T_0}
        \\
        b_\kth &= \iu \inParentheses{c_\kth - c_{-\kth}} = \frac{2 \, \beta}{T_0}
    \end{aligned}
    \right.
\]

Esto no es casualidad y viene dado por una propiedad que sale de sumar complejos conjugados.
Podemos comprobarla gráficamente.
Dado que $a_\kth \in \setR$ y $b_\kth \in \setI$ son ortogonales entre sí, los coeficientes $c_{\pm\kth} \in \setC$ no forman cualquier ángulo, sino que $c_\kth$ es el conjugado de $c_{-\kth}$.
A continuación se muestran con flechas en el plano complejo:

\begin{center}
    \def\svgwidth{0.6\linewidth}
    \input{./images/fourier-abc.pdf_tex}
\end{center}

A partir del gráfico anterior, se define el complejo $C_\kth = a_\kth - \iu \, b_\kth$ representado por un punto en el vértice del triángulo gris.

Luego, multiplicando y dividiendo por $\norm{C_\kth}$ la serie de la definición \ref{defn:FourierSerieTrig} se tiene
\[
    \fx[x]{t} = c_0 + \sum_{\kth=1}^{\infty} \norm{C_\kth} \inBrackets{ \dfrac{a_\kth}{\norm{C_\kth}} \fx[\cos]{\kth \, \omega_0 \, t} + \dfrac{b_\kth}{\norm{C_\kth}} \fx[\sin]{\kth \, \omega_0 \, t} }
\]

Por trigonometría se tiene $\fx[\cos]{\theta_\kth} = a_\kth / \norm{C_\kth}$ mientras que $\fx[\sin]{\theta_\kth} = b_\kth / \norm{C_\kth}$ luego:
\[
    \fx[x]{t} = c_0 + \sum_{\kth=1}^{\infty} \norm{C_\kth} \inBrackets{ \fx[\cos]{\theta_\kth} \fx[\cos]{\kth \, \omega_0 \, t} + \fx[\sin]{\theta_\kth} \fx[\sin]{\kth \, \omega_0 \, t} }
\]

Aplicando $\fx[\cos]{\alpha \pm \beta} = \fx[\cos]{\alpha} \fx[\cos]{\beta} \mp \fx[\sin]{\alpha} \fx[\sin]{\beta}$ se infiere:

\begin{mdframed}[style=DefinitionFrame]
    \begin{defn}
        \label{defn:FourierSerieArm}
    \end{defn}
    \cusTi{Serie de Fourier: forma armónica}
    \[
        \fx[x]{t} = c_0 + \sum_{\kth=1}^{\infty} \norm{C_\kth} \fx[\cos]{\kth \, \omega_0 \, t - \theta_\kth}
    \]
    \noTi{donde $C_\kth$ son los coeficientes de Fourier expresados en forma polar, tal que:}
    \[
    \left\{
    \begin{aligned}
        \norm{C_\kth} = \sqrt{a_\kth^2 + b_\kth^2}
        \\
        \theta_\kth = \fx[\artan]{\frac{b_\kth}{a_\kth}}
    \end{aligned}
    \right.
\]
\end{mdframed}

\begin{mdframed}[style=ExampleFrame]
    \begin{example}
    \end{example}
    \cusTi{De la proyección ortogonal a la serie}
    \begin{formatI}
        Calcular la proyección ortogonal de una onda cuadrada.
        Obtener los coeficientes de Fourier complejos.
        Demostrar que son equivalentes a una serie deducida a partir de la proyección.
    \end{formatI}
    \vspace{1em}
    Sea $x$ una función por partes dada por
    \[
        \fx[x]{t} =
        \left\{
        \begin{matrix}
            0 & \text{si } -\pi < t < -\frac{\pi}{2}
            \\
            1 & \text{si } -\frac{\pi}{2} < t < \frac{\pi}{2}
            \\
            0 & \text{si } \frac{\pi}{2} < t < \pi
        \end{matrix}
        \right.
    \]
    
    Sea $B_S$ una base de polinomios trigonométricos.
    A saber:
    \[
        B_S = \inBraces{1; \fx[\sin]{t}; \fx[\cos]{t}; \fx[\sin]{2t}; \dots; \fx[\sin]{\kth t}; \fx[\cos]{\kth t}}
    \]
    
    De manera que la proyección de $x$ sobre $S$ es:
    \begin{multline*}
        \fx[\proy_S]{x} =
        \\
        = \frac{1}{2} + \frac{2}{\pi} \inBrackets{\fx[\cos]{t} - \dfrac{\fx[\cos]{3t}}{3} + \dfrac{\fx[\cos]{5t}}{5} - \dfrac{\fx[\cos]{7t}}{7} \cdots}
    \end{multline*}

    Y puede ser expresada por la siguiente serie:
    \begin{equation}
        \fx[\proy_S]{x} = \frac{1}{2} + \frac{2}{\pi} \sum_{\kth=1}^\infty \frac{\inParentheses{-1}^{\kth+1}}{2 \, \kth - 1} \, \fx[\cos]{\inParentheses{2 \, \kth -1} t}
        \label{eqn:FourierProy}
    \end{equation}

    Para un pulso rectangular genérico
    \[
        \fx[x]{t} = A \, \fx[\Pi]{\frac{t-t_0}{\tau}}
    \]
    
    los coeficientes están dados según la definición \ref{defn:FourierSerieExp} como sigue:
    \begin{align*}
        c_\kth &= \frac{1}{T_0} \int_{T_0} \fx[x]{t} \, e^{-\iu \, \kth \, \omega_0 \, t} \, \dif t
        \\[1ex]
        &= \frac{A}{T_0} \int_{-\frac{\tau}{2}}^{\frac{\tau}{2}} e^{-\iu \, \kth \, \omega_0 \, t} \, \dif t
        \\[1ex]
        &= \frac{A}{T_0} \barrow{\dfrac{e^{-\iu \, \kth \, \omega_0 \, t}}{-\iu \, \kth \, \omega_0}}{-\frac{\tau}{2}}{\frac{\tau}{2}}
        \\[1ex]
        &= \frac{A}{-2 \, \pi \, \iu \, \kth } \inParentheses{ e^{ -\iu \, \omega_0 \, \kth \frac{\tau}{2}} - e^{ \iu \, \omega_0 \, \kth \frac{\tau}{2}} }
        \\[1ex]
        &= \frac{A}{\pi \, \kth} \, \fx[\sin]{\frac{\omega_0 \, \kth \, \tau}{2}}
        \\[1ex]
        &= \frac{A}{\pi \, \kth} \, \fx[\sin]{\frac{\pi \, \kth \, \tau}{T_0}}
        \\[1ex]
        &= \frac{A \, \tau}{T_0} \cdot
        \frac{ \fx[\sin]{\dfrac{\pi \, \kth \, \tau}{T_0}} }{ \dfrac{\pi \, \kth \, \tau}{T_0} }
    \end{align*}

    Esto es
    \[
        c_\kth = \frac{A \, \tau}{T_0} \, \fx[\sinc]{\frac{\pi \, \kth \, \tau}{T_0}}
    \]

    Particularmente si $A=1$, $T_0=2\,\pi$, $\tau=\pi$ queda
    \[
        c_\kth = \frac{ \fx[\sin]{\frac{\kth \, \pi}{2}} }{\kth \, \pi}
    \]

    Según la ecuación \ref{eqn:FourierExpBis} se tiene:
    \begin{align*}
        \fx[x]{t} &= c_0 + \sum_{\kth=1}^{\infty} \inParentheses{ c_\kth \, e^{\iu \, \kth \, \omega_0 \, t} + c_{-\kth} \, e^{-\iu \, \kth \, \omega_0 \, t} }
        \\
        &= c_0 + \sum_{\kth=1}^{\infty} \frac{\fx[\sin]{\frac{\kth \, \pi}{2}}}{\kth \, \pi} \inParentheses{ e^{\iu \, \kth \, \omega_0 \, t} + e^{-\iu \, \kth \, \omega_0 \, t} }
    \end{align*}

    Obteniendo el desarrollo en series de Fourier para una onda cuadrada:
    \[
        \fx[x]{t} = c_0 + 2 \sum_{\kth=1}^{\infty} \frac{\fx[\sin]{\frac{\kth \, \pi}{2}}}{\kth \, \pi} \, \fx[\cos]{\kth \, \omega_0 \, t}
    \]

    Observar que este desarrollo es equivalente a la ecuación \ref{eqn:FourierProy}, ya que $\fx[\sin]{\frac{\kth \, \pi}{2}}$ vale 0 para los $\kth$ pares y $\pm1$ para los $\kth$ impares.
\end{mdframed}

\begin{mdframed}[style=PropertyFrame]
    \begin{prop}
    \end{prop}
    Los coeficientes $c_\kth^\nth$ de la serie de Fourier de la n-ésima derivada están dados por
    \[
        c_\kth^\nth = c_\kth \cdot \inParentheses{\frac{\iu \, 2 \, \pi \, \kth}{T_0}}^\nth
    \]
\end{mdframed}

\section{La finalidad de la exponencial compleja}

Para una función $\fx[x]{t}$ real, el coeficiente complejo $c_0$ resulta ser el valor promedio de la misma:
\[
    c_0 = \frac{1}{T_0} \int_{T_0} \fx[x]{t} \, \dif t
\]

Si se reemplaza $\fx[x]{t}$ por su desarrollo en serie (Def. \ref{defn:FourierSerieExp}), se obtiene
\begin{align}
    c_0
    &= \frac{1}{T_0} \int_{T_0} \inParentheses{\sum_{\kth=-\infty}^\infty c_\kth \, e^{\iu \, \kth \, \omega_0 \, t}} \dif t
    \notag
    \\[1ex]
    &= \frac{1}{T_0} \int_{T_0} \inBrackets{\dots c_{-1} \, e^{\iu \inParentheses{-1} \, \omega_0 \, t} + c_0 \, e^0 + c_1 \, \, e^{\iu \inParentheses{1} \, \omega_0 \, t} \dots} \dif t
    \label{eqn:coeficientes}
    \\[1ex]
    &= \cdots \frac{1}{T_0} \int_{T_0} c_{-1} \, e^{- \iu \, \omega_0 \, t} \dif t
    + c_0
    + \frac{1}{T_0} \int_{T_0} c_1 \, \, e^{\iu \omega_0 \, t} \dif t \dots
    \notag
\end{align}
poniéndose en evidencia que la única integral no nula es
\[
    \frac{1}{T_0} \int_{T_0} c_0 \, e^0 \, \dif t = c_0
\]
ya que las demás son integrales de funciones periódicas evaluadas en un período, como por ejemplo lo sería
\[
    \int_0^{2\pi} \fx[\sin]{t} \, \dif t = 0
\]

Multiplicando a la ecuación \ref{eqn:coeficientes} el término $e^{-\iu \, \kth \, \omega_0 \, t}$ se consigue anular todos los sumandos quedando solamente el $k$-ésimo.
Esto se observa aplicando distributiva:
\begin{align*}
    c_\kth &=
    \scale{0.95}{
    \frac{1}{T_0} \int_{T_0} \inBrackets{\dots c_{-1} \, e^{-\iu \, \omega_0 \, t} + c_0 + c_1 \, \, e^{\iu \, \omega_0 \, t} \dots} e^{-\iu \, \kth \, \omega_0 \, t} \dif t
    }
    \\[1ex]
    &= 
    \scale{0.95}{
    \frac{1}{T_0} \int_{T_0} \inBrackets{\dots c_{-1} \, e^{-\iu \, \omega_0 \, t \inParentheses{1 + k}} + c_0 \, e^{-\iu \, \kth \, \omega_0 \, t} + c_1 \, \, e^{\iu \, \omega_0 \, t \inParentheses{1 - k}} \dots} \dif t
    }
\end{align*}

De esta forma, podemos verificar que para cualquier $k$ dado, el respectivo término exponencial se simplifica quedando
\[
    \frac{1}{T_0} \int_{T_0} c_k \, e^0 \, \dif t = c_k
\]
mientras que todos los otros términos se anulan como se mencionó anteriormente.

Formalmente, entendemos que los coeficientes de Fourier están definidos como tal por ser el producto interno entre la función que se está aproximando y un subespacio de funciones integrables.

\section{Transformada de Fourier}

La transformada de Fourier extiende el concepto de serie de Fourier a funciones no periódicas.
Permite descomponer señales en una superposición continua de exponenciales complejas de diferentes frecuencias.

Para inferir la transformada de Fourier de manera intuitiva, es preciso considerar en la definición \ref{defn:FourierSerieExp} el siguiente cambio de variable
\[
    \omega = \kth \, \omega_0
\]
tanto para el desarrollo en serie de Fourier
\begin{equation}
    \fx[x]{t}
    = \sum_{\kth=-\infty}^\infty
    c_\kth \, \complexExp
    \label{eqn:serieFourierOmega}
\end{equation}
como para la expresión de los coeficientes
\[
    c_\kth =
    \frac{1}{T_0}
    \int_{-\frac{T_0}{2}}^{\frac{T_0}{2}} \fx[x]{t} \, e^{- \iu \omega t}
    \, \dif t
\]

Pero queremos contemplar la posibilidad de que la función $\fx[x]{t}$ a desarrollar sea de duración finita, extendible por cero fuera de su dominio, pero sin tener que ser periódica.
Además, $\omega$ debería poder tomar cualquier valor real, mientras que a partir de $k \, \omega_0$ solo es posible obtener los armónicos, que son frecuencias discretas.
Para sortear ambas cuestiones, podemos asumir que el período de $\fx[x]{t}$ es
\[
    T_0 \to \infty
\]

Aplicando este límite, se tiene
\begin{equation}
    \lim_{T_0 \to \infty} c_\kth =
    \frac{\omega_0}{2 \pi}
    \int_{-\infty}^\infty \fx[x]{t} \, e^{- \iu \omega t}
    \, \dif t
    \quad \text{con } \omega_0 \to 0
    \label{eqn:coefFourierOmega}
\end{equation}

Quedando definido el \emph{espectro frecuencial} como el límite de los coeficientes de Fourier escalados

\begin{mdframed}[style=DefinitionFrame]
    \begin{defn}
    \end{defn}
    \cusTi{Espectro frecuencial}
    \[
        \fx[X]{\omega} = \lim_{T_0 \to \infty} T_0 \, c_k
    \]
\end{mdframed}

Su expresión está definida por la transformada de Fourier:

\begin{mdframed}[style=DefinitionFrame]
    \begin{defn}
        \label{defn:FourierTrans}
    \end{defn}
    \cusTi{Transformada de Fourier}
    \[
        \fourier{x}
        = \int_{-\infty}^\infty \fx[x]{t} \, e^{- \iu \omega t} \, \dif t
        = \fx[X]{\omega}
    \]
\end{mdframed}

Al reemplazar la expresión obtenida para los coeficientes de la ecuación \ref{eqn:coefFourierOmega} en el desarrollo en series de Fourier de la ecuación \ref{eqn:serieFourierOmega}, se tiene:
\[
    \fx[x]{t}
    = \sum_{\kth=-\infty}^\infty
    \inBrackets{ \frac{\omega_0}{2 \pi} \int_{-\infty}^{\infty} \fx[x]{\tau} \, e^{- \iu \omega \tau} \, \dif \tau }
    \, \complexExp
    \quad \text{con } \omega_0 \to 0
\]

Cuando $T_0 \to \infty$, el espaciado entre frecuencias $\omega_0 = \frac{2\pi}{T_0}$ tiende a cero.
La suma sobre frecuencias discretas $\omega = k \, \omega_0$ se aproxima a una integral sobre el dominio continuo $\omega \in \setR$.
Y el paso $\omega_0$ juega el papel de $\dif \omega$ en el límite de Riemann, que da origen a la transformada de Fourier inversa.

\begin{mdframed}[style=DefinitionFrame]
    \begin{defn}
        \label{defn:FourierTransInv}
    \end{defn}
    \cusTi{Transformada de Fourier inversa}
    \[
        \ffourier{X}
        = \frac{1}{2 \, \pi} \int_{-\infty}^\infty \fx[X]{\omega} \, \complexExp \, \dif \omega
        = \fx[x]{t}
    \]
\end{mdframed}
