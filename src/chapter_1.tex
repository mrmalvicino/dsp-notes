\chapter{Señales}

Se denomina \emph{señal} a la variación de cierta magnitud física que se utiliza para transmitir información.

\begin{itemize}
    \item
    Una señal $\fx[x]{t}$ es \textbf{continua} si está definida para todo instante de tiempo $t \in \setR$.
    Si además puede tomar cualquier valor $x \in \setR$, se la llama \textbf{analógica}.

    \item
    Una señal $\fx[x]{n}$ es \textbf{discreta} si está definida solo para ciertas muestras $n \in \setZ$.
    Si es discreta en su dominio y además toma valores $x$ en un conjunto cuantizado, se la llama \textbf{digital}.
\end{itemize}

Una señal continua puede ser transformada en discreta mediante un proceso de muestreo.
La \textbf{frecuencia de muestreo}, denotada como $f_s$, indica cuántas muestras por segundo se extraen en el proceso.
La señal $\fx[x]{n}$ resultante representaría una secuencia de valores obtenida al ventanear $\fx[x]{t}$ tal que $n = f_s \, t$.

\section{Período de señales discretas}

\begin{mdframed}[style=DefinitionFrame]
    \begin{defn}
        \label{defn:funcDiscPeriod}
    \end{defn}
    \cusTi{Función discreta periódica}
    \cusTe{$x: \setZ \longrightarrow \setC$ es una función periódica si y solo si}
    \begin{equation*}
        x[n] = x[n+N]
    \end{equation*}
    \noTi{donde $N \in \setN$ es el período.}
\end{mdframed}

Sea $\Omega \in \setR$ la frecuencia angular, por definición se tiene
\begin{gather*}
    \left\{
    \begin{aligned}
        & x[n] = e^{\iu \Omega n}
        \\
        & x[n+N] = e^{\iu \Omega n} \, e^{\iu \Omega N}
    \end{aligned}
    \right.
    \\[1ex]
    e^{\iu \Omega n} = e^{\iu \Omega n} \, e^{\iu \Omega N}
    \\
    1 = e^{\iu \Omega N}
\end{gather*}

Y 1 puede ser expresado como un número complejo en su forma polar, quedando luego:
\begin{gather*}
    e^{\iu 2 k \pi} = e^{\iu \Omega N} \text{ con $k \in \setN$}
    \\
    \iu \, 2 \, k \, \pi = \iu \, \Omega \, N
    \\
    N = \frac{2 \, k \, \pi}{\Omega}
\end{gather*}

Pudiendo definir para $k=1$ el período fundamental
\begin{equation*}
    N_0 = \frac{2 \, \pi}{\Omega}
\end{equation*}

Observar que $2 \, k \in \setN \, \forall \, k$ pero, dado que se tiene que cumplir que $N \in \setN$ y $\Omega \in \setR$, para que una función discreta sea periódica se tiene que verificar que $\Omega$ sea múltiplo de $\pi$.

\begin{mdframed}[style=PropertyFrame]
    \begin{prop}
    \end{prop}
    \begin{equation*}
        x[n] \text{ es periódica} \iff \frac{\pi}{\Omega} \in \setN
    \end{equation*}
\end{mdframed}

Además, si $N=1$ fuese un período válido, la función tomaría el mismo valor en todas las muestras.
Y en tal caso $x[n] = x[n_\ith]$ para todo $n_\ith$ mientras que la implicancia en la definición \ref{defn:funcDiscPeriod} es en ambos sentidos.
Con lo cual, se deduce que:

\begin{mdframed}[style=PropertyFrame]
    \begin{prop}
    \end{prop}
    \noTi{El mínimo período es $N=2$ con lo cual $\Omega_0 = \pi$ es la máxima frecuencia.}
\end{mdframed}

\section{Señales simples}

\concept{Escalón unitario}

\begin{equation*}
    \fx[u]{t} =
    \left\{
    \begin{aligned}
        0 \text{ si $t<0$}
        \\
        1 \text{ si $t>0$}
    \end{aligned}
    \right.
\end{equation*}

En $t=0$ puede estar definida como 0, 1 o 1/2 según la aplicación.

\begin{center}
    \def\svgwidth{0.6\linewidth}
    \input{./images/signal-unit-step.pdf_tex}
\end{center}

\concept{Signo}

\begin{equation*}
    \fx[\sgn]{t} =
    \left\{
    \begin{aligned}
        -1 \text{ si $t<0$}
        \\
        1 \text{ si $t>0$}
    \end{aligned}
    \right.
\end{equation*}

En $t=0$ puede estar definida como -1, 1 o 0 según la aplicación.

\begin{center}
    \def\svgwidth{0.6\linewidth}
    \input{./images/signal-sgn.pdf_tex}
\end{center}

El escalón unitario y la función signo son \emph{funciones escalonadas}.
Ambas se relacionan de la siguiente forma.

\begin{mdframed}[style=PropertyFrame]
    \begin{prop}
    \end{prop}
    \begin{equation*}
        \fx[u]{t} = \frac{\fx[\sgn]{t} + 1}{2}
    \end{equation*}
\end{mdframed}

\concept{Rampa unitaria}

Se suele denotar como $\fx[r]{t}$ y representa una señal que aumenta a tasa constante.

\begin{center}
    \def\svgwidth{0.6\linewidth}
    \input{./images/signal-unit-ramp.pdf_tex}
\end{center}

\concept{Tren de pulsos rectangulares y onda cuadrada}

\begin{equation*}
    \fx[x]{t} = A \, \pulse[t_0]{t}{\tau}
\end{equation*}

\begin{center}
    \def\svgwidth{0.8\linewidth}
    \input{./images/signal-sq-wave.pdf_tex}
\end{center}

El valor pico es representado por $A$ ya que eventualmente vale 1 (como en el escalón unitario) pero puede estar escalada.
También pueden darse corrimientos en tiempo según $t_0$ o desplazamientos del valor medio que la señal tome, en el eje vertical.

La relación entre el ancho ($\tau$) de los pulsos y el período ($T$) de la señal es llamada \emph{duty cycle}.
En porcentajes se expresa:
\begin{equation*}
    \text{duty cycle}  \equiv \frac{\tau}{T} \cdot 100\%
\end{equation*}

Si esta relación es de 1:2 entonces el duty cycle es de $50\%$ y en tal caso el tren de pulsos se llama \emph{onda cuadrada}.

\concept{Tren de pulsos triangulares y onda triangular}

\begin{equation*}
    \fx[x]{t} = A - \norm{t-t_0}
\end{equation*}

\begin{center}
    \def\svgwidth{0.8\linewidth}
    \input{./images/signal-triang-wave.pdf_tex}
\end{center}


\section{Funciones generalizadas}

El escalón unitario puede ser interpretado como la derivada de la rampa unitaria.
Esto pareciera no tener sentido, porque la pendiente de la rampa pasa de ser nula a ser unitaria de manera no suave.
Y efectivamente, en el sentido tradicional de la definición de derivada sería una interpretación errónea.
Por este motivo el valor que el escalón toma en $t=0$ es arbitrario.
Pero de igual manera es posible definir una función que sea la derivada del escalón unitario.
En cero la función pasa de valer 0 a valer 1 abruptamente y no es posible definir la derivada en el sentido tradicional.
Pero es posible dar una noción de derivada uniendo los dos tramos del escalón con una rampa que va ``desde la izquierda del cero hasta la derecha del cero'' para obtener una función continua.

\begin{center}
    \def\svgwidth{0.6\linewidth}
    \input{./images/signal-impulse-2.pdf_tex}
\end{center}

Personalmente, me gusta pensar que se trata de una función que tiene 2 dimensiones temporales: una en el sentido de $t$ y otra hacia ``adentro''.
De manera que es posible que la función tenga un cambio ``suave'' y discontinuo simultáneamente para un único instante $t_0$.
Así, el reloj no estaría frenándose sino que estaría usando otras manecillas.

Vemos que cuando el ``corte en cero'' tiende a unirse, la rampa que une ambos tramos es una línea vertical con pendiente infinita.
Este razonamiento sale de entender las señales como \emph{funciones generalizadas}.

Las funciones generalizadas son objetos matemáticos que pretenden extender el concepto tradicional de función, para poder definir una noción de derivada para funciones que no son derivables.
Fueron introducidas por Serguéi Sóbolev en 1935 y formalizadas en 1950 por Laurent Schwartz en su \emph{teoría de distribuciones}.

Las funciones tradicionales se definen con una asignación entre conjuntos y una fórmula, y el cálculo infinitesimal está dado por la integral de Riemann.
En cambio, \textbf{a muy grandes rasgos}, una función generalizada se define como una función partida según el entorno de un punto.
El estudio infinitesimal se hace a partir de un \emph{funcional}, para luego definir la función generalizada a partir del \emph{límite débil} del funcional.

\begin{mdframed}[style=PropertyFrame]
    \begin{prop}
        \label{prop:limDebil}
    \end{prop}
    \cusTi{Límite débil de una distribución}
    \cusTe{Para un funcional $\fx{x}$ y una función continua de prueba $\fx[\phi]{x}$ se cumple que}
    \begin{equation*}
        \lim_{x \to 0} \int \fx{x-x_0} \, \fx[\phi]{x} \, \dif x = \fx[\phi]{x_0}
    \end{equation*}
\end{mdframed}

\concept{Delta de Dirac e impulso unitario}

Se suele denotar como
\begin{equation*}
    \fx[\delta]{t-t_0} =
    \left\{
    \begin{aligned}
        \infty \text{ si $t=t_0$ }
        \\
        0 \text{ si $t \neq t_0$ }
    \end{aligned}
    \right.
\end{equation*}

Y se la representa gráficamente con una flecha haciendo referencia a que en el punto $t_0$ la función toma, por un instante, un valor infinito.
La \emph{altura} de la flecha indica el área ($\lambda$).
Si la función tiene área unitaria, se la llama impulso unitario.

\begin{center}
    \def\svgwidth{0.6\linewidth}
    \input{./images/signal-impulse-1.pdf_tex}
\end{center}

Los cuadrados grises representan el área de un pulso centrado en $t_0$, inicialmente de ancho $\tau$ y amplitud $\lambda / \tau$.
El área es el producto de la base por la altura, dada en este caso por $\lambda$.
Al tomar el límite $\tau \to 0$ vemos que el pulso tiende a ser un impulso.
El ancho tiende a ser nulo y la altura infinita pero manteniéndose el área constante.

Esto se debe a que $\frac{\lambda}{\tau} \, \pulse[t_0]{t}{\tau}$ es el funcional tal que
\begin{equation*}
    \lim_{\tau \to 0} \frac{\lambda}{\tau} \, \pulse[t_0]{t}{\tau} = \lambda \, \fx[\delta]{t-t_0}
\end{equation*}

Al integrar, según la propiedad \ref{prop:limDebil} se tiene
\begin{equation*}
    \lim_{\tau \to 0} \int_{-\infty}^\infty \frac{\lambda}{\tau} \, \pulse[t_0]{t}{\tau} \, \dif t
    = \int_{-\infty}^\infty \lambda \, \fx[\delta]{t-t_0} \, \dif t
    = \lambda
\end{equation*}

Para el caso del impulso unitario, donde $\lambda = 1$, queda:

\begin{mdframed}[style=PropertyFrame]
    \begin{prop}
    \end{prop}
    \begin{equation*}
        \int_{-\infty}^\infty \fx[\delta]{t} \, \dif t = 1
    \end{equation*}
\end{mdframed}

\section{Energía y potencia de señales}

\begin{mdframed}[style=DefinitionFrame]
    \begin{defn}
    \end{defn}
    \cusTi{Energía de una señal continua}
    \begin{equation*}
        E = \int_{-\infty}^{\infty} \norm{\fx[x]{t}}^2 \dif t
    \end{equation*}
\end{mdframed}

\begin{mdframed}[style=DefinitionFrame]
    \begin{defn}
    \end{defn}
    \cusTi{Potencia de una señal continua}
    \begin{equation*}
        P = \lim_{T \to \infty} \frac{1}{2T} \int_{-T}^{T} \norm{\fx[x]{t}}^2 \dif t
    \end{equation*}
\end{mdframed}

\begin{mdframed}[style=DefinitionFrame]
    \begin{defn}
    \end{defn}
    \cusTi{Energía de una señal discreta}
    \begin{equation*}
        E = \sum_{n=-\infty}^{\infty} \norm{\fx[x]{n}}^2
    \end{equation*}
\end{mdframed}

La potencia de una señal discreta tiene una pequeña diferencia con su contraparte continua.
Es necesario dividir por $2N+1$ ya que el promedio incluye la muestra $n=0$.

\begin{mdframed}[style=DefinitionFrame]
    \begin{defn}
    \end{defn}
    \cusTi{Potencia de una señal discreta}
    \begin{equation*}
        P = \lim_{N \to \infty} \inBrackets{\displaystyle \frac{1}{2N+1} \displaystyle \sum_{n=-N}^{N} \norm{\fx[x]{n}}^2}
    \end{equation*}
\end{mdframed}

\begin{mdframed}[style=PropertyFrame]
    \begin{prop}
    \end{prop}
    \begin{equation*}
        \sub{E}{tot} \text{ converge} \implies \sub{P}{tot} \to 0
    \end{equation*}
\end{mdframed}

\begin{mdframed}[style=PropertyFrame]
    \begin{prop}
    \end{prop}
    \begin{equation*}
        \sub{P}{tot} \text{ converge} \implies \sub{E}{tot} \to \infty
    \end{equation*}
\end{mdframed}

\begin{mdframed}[style=PropertyFrame]
    \begin{prop}
    \end{prop}
    \begin{equation*}
        \sub{P}{tot} \to \infty \implies \sub{E}{tot} \to \infty
    \end{equation*}
\end{mdframed}