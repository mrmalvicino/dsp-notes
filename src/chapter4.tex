\chapter{Transformada de Fourier}


\section{Serie de Fourier}

Una función $\fx[x]{t}$ de período fundamental $T_0 = \frac{2 \, \pi}{\omega_0}$ tiene desarrollo de Fourier si cumple las \emph{condiciones de Dirichlet}:
\begin{itemize}
    \item Es absolutamente inegrable: $\int_{T_0} \norm{\fx[x]{t}} \, \dif t < \infty$
    \item Tiene una cantidad finita de máximos y mínimos en $T_0$
    \item Tiene una cantidad finita de discontinuidades en $T_0$
\end{itemize}

\begin{mdframed}[style=MyFrame1]
    \begin{defn}
        \label{defn:FourierSerieExp}
    \end{defn}
    \cusTi{Serie de Fourier: forma exponencial compleja}
    \begin{equation*}
        \fx[x]{t} = \sum_{\kth=-\infty}^\infty c_\kth \, e^{\iu \, \kth \, \omega_0 \, t}
    \end{equation*}
    \noTi{donde $c_\kth$ son los coeficientes de Fourier complejos dados por}
    \begin{equation*}
        c_\kth = \frac{1}{T_0} \int_{T_0} \fx[x]{t} \, e^{-\iu \, \kth \, \omega_0 \, t} \, \dif t
    \end{equation*}
\end{mdframed}

Reacomodando la sumatoria de la definición \ref{defn:FourierSerieExp} como sigue
\begin{align*}
    \fx[x]{t} &=
    \sum_{\kth=-\infty}^\infty c_\kth \, e^{\iu \, \kth \, \omega_0 \, t}
    \\[1em]
    &= \sum_{\kth=-\infty}^{-1} c_\kth \, e^{\iu \, \kth \, \omega_0 \, t} + c_0 + \sum_{\kth=1}^{\infty} c_\kth \, e^{\iu \, \kth \, \omega_0 \, t}
    \\[1em]
    &= \sum_{\kth=1}^{\infty} c_{-\kth} \, e^{-\iu \, \kth \, \omega_0 \, t} + c_0 + \sum_{\kth=1}^{\infty} c_\kth \, e^{\iu \, \kth \, \omega_0 \, t}
\end{align*}

es posible obtener la siguiente expresión equivalente:
\begin{equation}
    \fx[x]{t} = c_0 + \sum_{\kth=1}^{\infty} \bb{ c_\kth \, e^{\iu \, \kth \, \omega_0 \, t} + c_{-\kth} \, e^{-\iu \, \kth \, \omega_0 \, t} }
    \label{eqn:FourierExpBis}
\end{equation}

Y aplicando la fórmula de Euler:
\begin{align*}
    &
    \scale{0.80}{
    = c_0 + \sum_{\kth=1}^{\infty} \braces{c_\kth \sqb{\fx[\cos]{\kth \, \omega_0 \, t} + \iu \fx[\sin]{\kth \, \omega_0 \, t}} + c_{-\kth} \sqb{\fx[\cos]{\kth \, \omega_0 \, t} - \iu \fx[\sin]{\kth \, \omega_0 \, t}} }
    }
    \\[1em]
    &= c_0 + \sum_{\kth=1}^{\infty} \sqb{ \bb{c_\kth + c_{-\kth}} \fx[\cos]{\kth \, \omega_0 \, t} + \iu \bb{c_\kth - c_{-\kth}} \fx[\sin]{\kth \, \omega_0 \, t} }
\end{align*}

\begin{mdframed}[style=MyFrame1]
    \begin{defn}
        \label{defn:FourierSerieTrig}
    \end{defn}
    \cusTi{Serie de Fourier: forma trigonométrica}
    \begin{equation*}
        \fx[x]{t} = c_0 + \sum_{\kth=1}^{\infty} \sqb{a_\kth \fx[\cos]{\kth \, \omega_0 \, t} + b_\kth \fx[\sin]{\kth \, \omega_0 \, t} }
    \end{equation*}
    \noTi{donde $a_\kth = c_\kth + c_{-\kth}$ y $b_\kth = \iu \bb{c_\kth - c_{-\kth}}$ son los coeficientes de Fourier trigonométricos.}
\end{mdframed}

Por otro lado, si se aplica la fórmula de Euler pero esta vez en la exponencial de los coeficientes complejos de la definición \ref{defn:FourierSerieExp}:
\begin{align*}
    c_{\pm\kth} &= \frac{1}{T_0} \int_{T_0} \fx[x]{t} \, e^{\mp \iu \, \kth \, \omega_0 \, t} \, \dif t
    \\
    &= \frac{1}{T_0} \int_{T_0} \fx[x]{t} \sqb{ \fx[\cos]{\kth \, \omega_0 \, t} \mp \iu \fx[\sin]{\kth \, \omega_0 \, t} } \, \dif t
    \\
    &= \frac{1}{T_0} \int_{T_0} \fx[x]{t} \fx[\cos]{\kth \, \omega_0 \, t} \dif t
    \mp \frac{\iu}{T_0} \int_{T_0} \fx[x]{t} \fx[\sin]{\kth \, \omega_0 \, t} \dif t
\end{align*}

Se definen los términos $\alpha = \int_{T_0} \fx[x]{t} \fx[\cos]{\kth \, \omega_0 \, t} \dif t$ y $\beta = \int_{T_0} \fx[x]{t} \fx[\sin]{\kth \, \omega_0 \, t} \dif t$ para simplificar la notación, tal que:
\begin{equation*}
     c_{\pm\kth} = \frac{\alpha}{T_0} \mp \iu \frac{\beta}{T_0}
\end{equation*}

Obteniendo las siguientes expresiones para los coeficientes trigonométricos de la definición \ref{defn:FourierSerieTrig}:
\begin{equation*}
    \left\{
    \begin{aligned}
        a_\kth &= c_\kth + c_{-\kth} = \frac{2 \, \alpha}{T_0}
        \\
        b_\kth &= \iu \bb{c_\kth - c_{-\kth}} = \frac{2 \, \beta}{T_0}
    \end{aligned}
    \right.
\end{equation*}

Esto no es casualidad y viene dado por una propiedad que sale de sumar complejos conjugados. Podemos comprobarla gráficamente. Dado que $a_\kth \in \setR$ y $b_\kth \in \setI$ son ortogonales entre sí, los coeficientes $c_{\pm\kth} \in \setC$ no forman cualquier ángulo, sino que $c_\kth$ es el conjugado de $c_{-\kth}$. A continuación se muestran con flechas en el plano complejo:

\begin{center}
    \def\svgwidth{0.6\linewidth}
    \input{./images/fourier-abc.pdf_tex}
\end{center}

A partir del gráfico anterior, se define el complejo $C_\kth = a_\kth - \iu \, b_\kth$ representado por un punto en el vértice del triángulo gris.

Luego, multiplicando y dividiendo por $\norm{C_\kth}$ la serie de la definición \ref{defn:FourierSerieTrig} se tiene
\begin{equation*}
    \fx[x]{t} = c_0 + \sum_{\kth=1}^{\infty} \norm{C_\kth} \sqb{ \dfrac{a_\kth}{\norm{C_\kth}} \fx[\cos]{\kth \, \omega_0 \, t} + \dfrac{b_\kth}{\norm{C_\kth}} \fx[\sin]{\kth \, \omega_0 \, t} }
\end{equation*}

Por trigonometría se tiene $\fx[\cos]{\theta_\kth} = a_\kth / \norm{C_\kth}$ mientras que $\fx[\sin]{\theta_\kth} = b_\kth / \norm{C_\kth}$ luego:
\begin{equation*}
    \fx[x]{t} = c_0 + \sum_{\kth=1}^{\infty} \norm{C_\kth} \sqb{ \fx[\cos]{\theta_\kth} \fx[\cos]{\kth \, \omega_0 \, t} + \fx[\sin]{\theta_\kth} \fx[\sin]{\kth \, \omega_0 \, t} }
\end{equation*}

Aplicando $\fx[\cos]{\alpha \pm \beta} = \fx[\cos]{\alpha} \fx[\cos]{\beta} \mp \fx[\sin]{\alpha} \fx[\sin]{\beta}$ se infiere:

\begin{mdframed}[style=MyFrame1]
    \begin{defn}
        \label{defn:FourierSerieArm}
    \end{defn}
    \cusTi{Serie de Fourier: forma armónica}
    \begin{equation*}
        \fx[x]{t} = c_0 + \sum_{\kth=1}^{\infty} \norm{C_\kth} \fx[\cos]{\kth \, \omega_0 \, t - \theta_\kth}
    \end{equation*}
    \noTi{donde $C_\kth$ son los coeficientes de Fourier expresados en forma polar, tal que:}
    \begin{equation*}
    \left\{
    \begin{aligned}
        \norm{C_\kth} = \sqrt{a_\kth^2 + b_\kth^2}
        \\
        \theta_\kth = \fx[\artan]{\frac{b_\kth}{a_\kth}}
    \end{aligned}
    \right.
\end{equation*}
\end{mdframed}

\begin{mdframed}[style=MyFrame2]
    \begin{example}
    \end{example}
    \cusTi{De la proyección ortogonal a la serie}
    \begin{formatI}
        Calcular la proyección ortogonal de una onda cuadrada. Obtener los coeficientes de Fourier complejos. Demostrar que son equivalentes a una serie deducida a partir de la proyección.
    \end{formatI}
    \vspace{1em}
    Sea $x$ una función por partes dada por
    \begin{equation*}
        \fx[x]{t} =
        \left\{
        \begin{matrix}
            0 & \text{si } -\pi < t < -\frac{\pi}{2}
            \\
            1 & \text{si } -\frac{\pi}{2} < t < \frac{\pi}{2}
            \\
            0 & \text{si } \frac{\pi}{2} < t < \pi
        \end{matrix}
        \right.
    \end{equation*}
    
    Sea $B_S$ una base de polinomios trigonométricos. A saber:
    \begin{equation*}
        B_S = \braces{1; \fx[\sin]{t}; \fx[\cos]{t}; \fx[\sin]{2t}; \dots; \fx[\sin]{\kth t}; \fx[\cos]{\kth t}}
    \end{equation*}
    
    De manera que la proyección de $x$ sobre $S$ es:
    \begin{multline*}
        \fx[\proy_S]{x} =
        \\
        = \frac{1}{2} + \frac{2}{\pi} \sqb{\fx[\cos]{t} - \dfrac{\fx[\cos]{3t}}{3} + \dfrac{\fx[\cos]{5t}}{5} - \dfrac{\fx[\cos]{7t}}{7} \cdots}
    \end{multline*}

    Y puede ser expresada por la siguiente serie:
    \begin{equation}
        \fx[\proy_S]{x} = \frac{1}{2} + \frac{2}{\pi} \sum_{\kth=1}^\infty \frac{\bb{-1}^{\kth+1}}{2 \, \kth - 1} \, \ffx[\cos]{\bb{2 \, \kth -1} t}
        \label{eqn:FourierProy}
    \end{equation}

    Para un pulso rectangular genérico
    \begin{equation*}
        \fx[x]{t} = A \, \fx[\Pi]{\frac{t-t_0}{\tau}}
    \end{equation*}
    
    los coeficientes están dados según la definición \ref{defn:FourierSerieExp} como sigue:
    \begin{align*}
        c_\kth &= \frac{1}{T_0} \int_{T_0} \fx[x]{t} \, e^{-\iu \, \kth \, \omega_0 \, t} \, \dif t
        \\[1ex]
        &= \frac{A}{T_0} \int_{-\frac{\tau}{2}}^{\frac{\tau}{2}} e^{-\iu \, \kth \, \omega_0 \, t} \, \dif t
        \\[1ex]
        &= \frac{A}{T_0} \barrow{\dfrac{e^{-\iu \, \kth \, \omega_0 \, t}}{-\iu \, \kth \, \omega_0}}{-\frac{\tau}{2}}{\frac{\tau}{2}}
        \\[1ex]
        &= \frac{A}{-2 \, \pi \, \iu \, \kth } \bb{ e^{ -\iu \, \omega_0 \, \kth \frac{\tau}{2}} - e^{ \iu \, \omega_0 \, \kth \frac{\tau}{2}} }
        \\[1ex]
        &= \frac{A}{\pi \, \kth} \, \fx[\sin]{\frac{\omega_0 \, \kth \, \tau}{2}}
        \\[1ex]
        &= \frac{A}{\pi \, \kth} \, \fx[\sin]{\frac{\pi \, \kth \, \tau}{T_0}}
        \\[1ex]
        &= \frac{A \, \tau}{T_0} \cdot
        \frac{ \fx[\sin]{\dfrac{\pi \, \kth \, \tau}{T_0}} }{ \dfrac{\pi \, \kth \, \tau}{T_0} }
    \end{align*}

    Esto es
    \begin{equation*}
        c_\kth = \frac{A \, \tau}{T_0} \, \fx[\sinc]{\frac{\pi \, \kth \, \tau}{T_0}}
    \end{equation*}

    Particularmente si $A=1$, $T_0=2\,\pi$, $\tau=\pi$ queda
    \begin{equation*}
        c_\kth = \frac{ \fx[\sin]{\frac{\kth \, \pi}{2}} }{\kth \, \pi}
    \end{equation*}

    Según la ecuación \ref{eqn:FourierExpBis} se tiene:
    \begin{align*}
        \fx[x]{t} &= c_0 + \sum_{\kth=1}^{\infty} \bb{ c_\kth \, e^{\iu \, \kth \, \omega_0 \, t} + c_{-\kth} \, e^{-\iu \, \kth \, \omega_0 \, t} }
        \\
        &= c_0 + \sum_{\kth=1}^{\infty} \frac{\fx[\sin]{\frac{\kth \, \pi}{2}}}{\kth \, \pi} \bb{ e^{\iu \, \kth \, \omega_0 \, t} + e^{-\iu \, \kth \, \omega_0 \, t} }
    \end{align*}

    Obteniendo el desarrollo en series de Fourier para una onda cuadrada:
    \begin{equation*}
        \fx[x]{t} = c_0 + 2 \sum_{\kth=1}^{\infty} \frac{\fx[\sin]{\frac{\kth \, \pi}{2}}}{\kth \, \pi} \, \fx[\cos]{\kth \, \omega_0 \, t}
    \end{equation*}

    Observar que este desarrollo es equivalente a la ecuación \ref{eqn:FourierProy}, ya que $\fx[\sin]{\frac{\kth \, \pi}{2}}$ vale 0 para los $\kth$ pares y $\pm1$ para los $\kth$ impares.
\end{mdframed}

\begin{mdframed}[style=MyFrame1]
    \begin{prop}
    \end{prop}
    Los coeficientes $c_\kth^\nth$ de la serie de Fourier de la n-ésima derivada están dados por
    \begin{equation*}
        c_\kth^\nth = c_\kth \cdot \bb{\frac{\iu \, 2 \, \pi \, \kth}{T_0}}^\nth
    \end{equation*}
\end{mdframed}


\section{Transformada de Fourier}

Para una función $\fx[x]{t}$ real, el coeficiente complejo $c_0$ resulta ser el valor promedio de la misma:
\begin{equation*}
    c_0 = \frac{1}{T_0} \int_{T_0} \fx[x]{t} \, \dif t
\end{equation*}

Si se reemplaza $\fx[x]{t}$ por su desarrollo en serie (Def. \ref{defn:FourierSerieExp}), se obtiene:
\begin{align}
    c_0 &= \frac{1}{T_0} \int_{T_0} \bb{\sum_{\kth=-\infty}^\infty c_\kth \, e^{\iu \, \kth \, \omega_0 \, t}} \dif t
    \notag
    \\[1ex]
    &= \frac{1}{T_0} \int_{T_0} \sqb{\dots c_{-1} \, e^{\iu \bb{-1} \, \omega_0 \, t} + c_0 \, e^0 + c_1 \, \, e^{\iu \bb{1} \, \omega_0 \, t} \dots} \dif t
    \label{eqn:coeficientes}
    \\[1ex]
    &= \cdots \frac{1}{T_0} \int_{T_0} c_{-1} \, e^{- \iu \, \omega_0 \, t} \dif t
    + c_0
    + \frac{1}{T_0} \int_{T_0} c_1 \, \, e^{\iu \omega_0 \, t} \dif t \dots
    \notag
\end{align}

Poniéndose en evidencia que la única integral no nula es $\frac{1}{T_0} \int_{T_0} c_0 \, e^0 \, \dif t = c_0$ ya que las demás son integrales de funciones periódicas evaluadas en un período, como por ejemplo lo sería $\int_0^{2\pi} \fx[\sin]{t} \, \dif t = 0$.

Multiplicando a la ecuación \ref{eqn:coeficientes} el término $e^{-\iu \, \kth \, \omega_0 \, t}$ se consigue anular todos los sumandos quedando solamente el $k$-ésimo. Esto se observa aplicando distributiva:
\begin{align*}
    c_\kth &=
    \scale{0.95}{
    \frac{1}{T_0} \int_{T_0} \sqb{\dots c_{-1} \, e^{-\iu \, \omega_0 \, t} + c_0 + c_1 \, \, e^{\iu \, \omega_0 \, t} \dots} e^{-\iu \, \kth \, \omega_0 \, t} \dif t
    }
    \\[1ex]
    &= 
    \scale{0.95}{
    \frac{1}{T_0} \int_{T_0} \sqb{\dots c_{-1} \, e^{-\iu \, \omega_0 \, t \bb{1 + k}} + c_0 \, e^{-\iu \, \kth \, \omega_0 \, t} + c_1 \, \, e^{\iu \, \omega_0 \, t \bb{1 - k}} \dots} \dif t
    }
\end{align*}

De esta forma, podemos ver conceptualmente que para cualquier $k$ dado, el respectivo término exponencial se simplifica quedando $\frac{1}{T_0} \int_{T_0} c_k \, e^0 \, \dif t = c_k$ mientras que todos los otros términos se anulan como se mencionó anteriormente.

Formalmente, entendermos que los coeficientes de Fourier están definidos como tal por ser el producto interno entre la función que se está aproximando y un subespacio de funciones integrables. Ver aproximación de señales para espacios euclideos en Notas de matemática para ingeniería de sonido.

Si en vez de usar el coeficiente $k$-ésimo definimos una función que devuelva cada coeficiente que modela la amplitud de cada frecuencia $\omega = k \, \omega_0$ obtenemos:
\begin{align*}
    c_\kth &= \frac{1}{T_0} \int_{T_0} \fx[x]{t} \, e^{-\iu \, \kth \, \omega_0 \, t} \, \dif t
    \\
    &= \frac{1}{T_0} \int_{T_0} \fx[x]{t} \, e^{-\iu \, \omega \, t} \, \dif t = \fx[c]{\omega}
\end{align*}

Cualquier función periódica puede ser desarrollada en series de Fourier y los coeficientes estarían dados según la ecuación anterior para frecuencias múltiplos de la fundamental, conocidas como armónicos. A saber:
\begin{equation*}
    \omega = k \, \omega_0
\end{equation*}

Pero la transformada de Fourier se define para funciones no necesariamente periódicas y, además, para un dominio continuo. Observar que $\omega$ debería poder tomar cualquier valor real, mientras que a partir de $k \, \omega_0$ solo es posible obtener frecuencias discretas.

Para sortear ambas cuestiones, partimos del desarrollo en serie de Fourier de una función cuyo período es infinitamente largo, con lo cual $T_0 \to \infty$. Observar que cualquier señal acotada en tiempo cumple esta hipótesis, por más que sea no periódica.
Así, haciendo el cambio de variable $\omega = k \, \omega_0$ y considerando $c_k = \fx[c]{\omega}$ en la ecuación \ref{defn:FourierSerieExp} se tiene
\begin{align*}
    \fx[x]{t} &= \sum_{\kth=-\infty}^\infty c_\kth \, e^{\iu \, \kth \, \omega_0 \, t}
    \\[1ex]
    &= \sum_{\kth=-\infty}^\infty \left[ \frac{1}{T_0} \int_{T_0} \fx[x]{t} \, e^{-\iu \, \omega \, t} \, \dif t \right] \, e^{\iu \, \omega \, t} \text{ con } T_0 \to \infty
    \\[1ex]
    &= \sum_{\kth=-\infty}^\infty \left[ \frac{\omega_0}{2 \, \pi} \int_{-\infty}^{\infty} \fx[x]{t} \, e^{-\iu \, \omega \, t} \, \dif t \right] \, e^{\iu \, \omega \, t} \text{ con } \omega_0 \to 0
\end{align*}

De manera que el espectro de frecuencias es equivalente a la transformada de Fourier y se define como sigue.

\begin{mdframed}[style=MyFrame1]
    \begin{defn}
    \end{defn}
    \cusTi{Espectro frecuencial}
    \begin{equation*}
        \fx[X]{\omega} = c_k \, T_0
    \end{equation*}
\end{mdframed}

\begin{mdframed}[style=MyFrame1]
    \begin{defn}
        \label{defn:FourierTrans}
    \end{defn}
    \cusTi{Transformada de Fourier}
    \begin{equation*}
        \fx[X]{\omega} = \int_{-\infty}^\infty \fx[x]{t} \, e^{-\iu \, \omega \, t} \, \dif t = \fourier{x}
    \end{equation*}
\end{mdframed}

Quedando el desarrollo en serie de la función dado por la antitransformada:
\begin{align*}
    \fx[x]{t} &= \sum_{\kth=-\infty}^\infty \left[ \frac{\omega_0}{2 \, \pi} \fx[X]{\omega} \right] \, e^{\iu \, \omega \, t} \text{ con } \omega_0 \to 0
    \\[1ex]
    &= \frac{1}{2 \, \pi} \sum_{\kth=-\infty}^\infty \fx[X]{\omega} \, e^{\iu \, \omega \, t} \, \omega_0 \text{ con } \omega_0 \to 0
\end{align*}

\begin{mdframed}[style=MyFrame1]
    \begin{defn}
        \label{defn:FourierTransInv}
    \end{defn}
    \cusTi{Transformada de Fourier inversa}
    \begin{equation*}
        \fx[x]{t} = \frac{1}{2 \, \pi} \int_{-\infty}^\infty \fx[X]{\omega} \, e^{\iu \, \omega \, t} \, \dif \omega = \ffourier{X}
    \end{equation*}
\end{mdframed}

\begin{mdframed}[style=MyFrame2]
    \begin{example}
    \end{example}
    \cusTi{Espectro bipolar y forma de onda}
    \begin{equation*}
        \fx[x]{t} = 5 \fx[\sin]{2 \, \pi \, \SI{2}{\hertz} \, t + \pi} + 7 \fx[\sin]{2 \, \pi \, \SI{3}{\hertz} \, t - \pi/2}
    \end{equation*}
    \begin{center}
        \def\svgwidth{0.6\linewidth}
        \input{./images/fourier-waveform.pdf_tex}
    \end{center}
    \begin{center}
        \def\svgwidth{\linewidth}
        \input{./images/fourier-freq-spec.pdf_tex}
    \end{center}
\end{mdframed}