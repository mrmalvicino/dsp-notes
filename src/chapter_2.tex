\chapter{Sistemas}

Un sistema es una operación que transforma una señal de entrada $x$ en una señal de salida $y$.
Puede modificar la señal de diversas maneras, según la naturaleza de la transformación.

\begin{mdframed}[style=DefinitionFrame]
    \begin{defn}
    \end{defn}
    \cusTi{Transformación asociada a un sistema}
    \cusTe{}
    \[
        \fx[y]{t} = \fx[T]{\fx[x]{t}}
    \]
\end{mdframed}

\begin{itemize}
    \item
    \textbf{Modificación de amplitud:}
    Puede escalar los valores de la señal, amplificándolos o atenuándolos.
    \item
    \textbf{Modificación en tiempo:}
    Puede desplazar la señal, invertirla temporalmente o comprimirla.
    \item
    \textbf{Filtrado:}
    Puede atenuar o eliminar ciertas componentes de frecuencia de la señal.
    \item
    \textbf{Integración y derivación:}
    Puede aplicar acumuladores o destacar cambios de una señal continua.
    \item
    \textbf{No lineales:}
    La distorción introduce frecuencias que no existían.
    La modulación cambia la forma de la señal.
\end{itemize}

\section{Clasificación de sistemas}

\begin{mdframed}[style=DefinitionFrame]
    \begin{defn}
    \end{defn}
    \cusTi{Sistema lineal}
    \cusTe{Un sistema es lineal si se puede aplicar el principio de superposición:}
    \[
        \fx[T]{a \, \fx[x_1]{t} + b \, \fx[x_2]{t}} = a \, \fx[y_1]{t} + b \, \fx[y_2]{t}
    \]
\end{mdframed}

\begin{mdframed}[style=DefinitionFrame]
    \begin{defn}
    \end{defn}
    \cusTi{Sistema invariante en el tiempo}
    \cusTe{Un sistema es invariante en el tiempo si su comportamiento no cambia con el tiempo.}
    \[
        \fx[T]{\fx[x]{t-t_0}} =\fx[y]{t-t_0}
    \]
\end{mdframed}

\begin{mdframed}[style=DefinitionFrame]
    \begin{defn}
    \end{defn}
    \cusTi{Sistema estable}
    \cusTe{Un sistema es estable si para una entrada acotada en amplitud, la salida también es acotada.}
\end{mdframed}

\begin{mdframed}[style=DefinitionFrame]
    \begin{defn}
    \end{defn}
    \cusTi{Sistema causal}
    \cusTe{Un sistema es causal o no anticipativo si la salida no comienza antes de aplicar la entrada.}
\end{mdframed}

\section{Función de transferencia}

La función de transferencia es una herramienta fundamental para el análisis de sistemas lineales e invariantes en el tiempo (LTI).
Describe cómo el sistema modifica las diferentes componentes de frecuencia de una señal.
Permite estudiar el comportamiento del sistema sin necesidad de resolver la ecuación diferencial que lo define.
Representa la relación entre la entrada y la salida del sistema en términos de frecuencia compleja.

\begin{mdframed}[style=DefinitionFrame]
    \begin{defn}
    \end{defn}
    \cusTi{Función de transferencia}
    \cusTe{Dado un sistema lineal e invariante en el tiempo se define la función de transferencia}
    \[
        \fx[H]{s} = \frac{\fx[Y]{s}}{\fx[X]{s}}
    \]
    donde $\fx[X]{s}$ e $\fx[Y]{s}$ son los espectros de las señales de entrada y salida, y $s = \sigma + \iu \omega$ la variable de Laplace.
\end{mdframed}

La parte real, $\sigma$, determina el crecimiento o decrecimiento exponencial.
Mientras que la parte imaginaria, $\omega$, determina la frecuencia angular de la señal.
Evaluar $\sigma = 0$ resulta útil porque permite estudiar cómo responde el sistema a señales senoidales.

Estudiando la respuesta cuando la entrada es $e^{\iu \omega \, t}$ es posible calcular $\fx[H]{\iu \omega}$.
Para este caso particular, la función de transferencia termina siendo un número complejo independiente del tiempo.
Y la expresión encontrada a partir de esta situación particular, es válida para cualquier señal de entrada.

\begin{mdframed}[style=PropertyFrame]
    \begin{prop}
    \end{prop}
    Si un sistema LTI es alimentado con una señal de entrada $\fx[x]{t} = e^{\iu \omega \, t}$ entonces
    \[
        \fx[H]{\iu \omega} = \frac{\fx[y]{t}}{\fx[x]{t}}
    \]
\end{mdframed}

\section{Espectro bilateral}

El espectro bilateral es una representación que muestra la distribución de las componentes de frecuencia de una señal.
Indica el aporte en amplitud y en fase de cada frecuencia para formar la totalidad de la señal.

En la realidad solo existen frecuencias mayores a cero.
Las señales reales pueden ser representadas por cosenos, que son funciones pares.
Por esto, el signo de su argumento (dado por el signo de la frecuencia) no afecta la representación.
\[
    \fx[\cos]{\omega \, t} = \fx[\cos]{- \omega \, t}
\]

Pero al usar modelos matemáticos que tienen un dominio complejo, es necesario considerar las frecuencias negativas.
\[
    e^{\iu \omega \, t} \neq e^{- \iu \omega \, t}
\]

Esto se evidencia con la siguiente ecuación, que permite representar una señal real como una suma de exponenciales complejas.
\[
    \fx[\cos]{\omega \, t} = \frac{e^{\iu \omega \, t} + e^{- \iu \omega \, t}}{2}
\]

Las señales reales pueden ser representadas por una suma de senoidales simples.
En ese desarrollo, eventualmente podría haber sumandos que sean senos o cosenos de amplitud negativa.
Pero por convención, es necesario reemplaar estos sumandos por el \textbf{coseno positivo} equivalente mediante un defasaje arbitrario.

\section{Simetría hermitiana}

Se tiene un sistema LIT que se exita con la señal
\[
    \fx[x]{t} = A_x \, \fx[\cos]{\omega_x \, t + \theta_x}
\]

Podemos asumir entonces, que la salida va a ser una señal senoidal de igual frecuencia.
Pero la amplitud y fase eventualmente se verán modificadas.
\[
    \fx[y]{t} = A_y \, \fx[\cos]{\omega_x \, t + \theta_y}
\]

Por ser la entrada una señal senoidal simple, la función de transferencia estará dada por
\[
    \fx[H]{\iu \omega} = \norm{\fx[H]{\iu \omega}} \, e^{\iu \, \fx[\arg]{\fx[H]{\iu \omega}}}
\]
tal que
\[
    \fx[y]{t} = \fx[H]{\iu \omega} \, \fx[x]{t}
\]

Sabiendo que
\[
    z + \conj{z} = 2 \, \norm{z} \fx[\cos]{\fx[\arg]{z}}
    = \norm{z} e^{\iu \fx[\arg]{z}} + \norm{z} e^{- \iu \fx[\arg]{z}}
\]
podemos escribir la señal de entrada como
\[
    \fx[x]{t} = \frac{A_x}{2} e^{\iu \inParentheses{\omega_x \, t + \theta_x}}
    + \frac{A_x}{2} e^{- \iu \inParentheses{\omega_x \, t + \theta_x}}
\]

Dado que el sistema es lineal, pordemos aplicar superposición.
Osea que la transformación a todo el espectro es equivalente a la transformación de cada frecuencia:
\begin{align*}
    \fx[y]{t} & = \sum_{\nth=1}^{2} \fx[H]{\iu \omega_\nth}
    \, \alpha_\nth
    \, e^{\iu \inParentheses{\omega_\nth t + \theta_\nth}}
    \\
    & = \sum_{\nth=1}^{2} \norm{\fx[H]{\iu \omega_\nth}}
    \, \alpha_\nth
    \, e^{\iu \inParentheses{\omega_\nth t + \theta_\nth + \fx[\arg]{\fx[H]{\iu \omega_\nth}}}}
\end{align*}

La señal de salida queda luego dada por
\begin{multline*}
    \fx[y]{t} = \norm{\fx[H]{\iu \omega_x}}
    \, \frac{A_x}{2}
    \, e^{\iu \inParentheses{\omega_x t + \theta_x + \fx[\arg]{\fx[H]{\iu \omega_x}}}}
    \\
    + \norm{\fx[H]{- \iu \omega_x}}
    \, \frac{A_x}{2}
    \, e^{\iu \inParentheses{- \omega_x t - \theta_x + \fx[\arg]{\fx[H]{- \iu \omega_x}}}}
\end{multline*}

Dado que la señal de entrada es real, la señal de salida debe tener parte imaginaria nula:
\begin{equation*}
    \fx[x]{t} \in \setR \implies \fx[\Im]{\fx[y]{t}} = 0
\end{equation*}

A saber, la parte imaginaria de un complejo es:
\begin{equation*}
    \fx[\Im]{z} = \fx[\Im]{\norm{z} \, e^{\iu \, \fx[\arg]{z}}}
    = \norm{z} \, \fx[\sin]{\fx[\arg]{z}}
\end{equation*}

Igualando la parte imaginaria de $\fx[y]{t}$ a cero, se obtiene la siguiente ecuación:
\begin{multline*}
    \norm{\fx[H]{\iu \omega_x}}
    \, \frac{A_x}{2}
    \, \fx[\sin]{\omega_x t + \theta_x + \fx[\arg]{\fx[H]{\iu \omega_x}}}
    =
    \\
    =
    - \norm{\fx[H]{- \iu \omega_x}}
    \, \frac{A_x}{2}
    \, \fx[\sin]{- \omega_x t - \theta_x + \fx[\arg]{\fx[H]{- \iu \omega_x}}}
\end{multline*}

Dado que $\fx[\sin]{-\varphi} = - \fx[\sin]{\varphi}$ se tiene:
\begin{multline*}
    \norm{\fx[H]{\iu \omega_x}}
    \, \fx[\sin]{\omega_x t + \theta_x + \fx[\arg]{\fx[H]{\iu \omega_x}}}
    =
    \\
    =
    \norm{\fx[H]{- \iu \omega_x}}
    \, \fx[\sin]{\omega_x t + \theta_x - \fx[\arg]{\fx[H]{- \iu \omega_x}}}
\end{multline*}

De manera que para que se cumpla la igualdad, el espectro de frecuencias tiene que tener amplitud par y fase impar.

\begin{mdframed}[style=PropertyFrame]
    \begin{prop}
    \end{prop}
    \cusTi{Simetría hermitiana}
    \cusTe{Un sistema lineal e invariante en el tiempo exitado por una señal senoidal simple de frecuencia $\omega_x$ verifica:}
    \begin{align*}
        \norm{\fx[H]{\iu \omega_x}} & = \norm{\fx[H]{- \iu \omega_x}}
        \\
        \fx[\arg]{\fx[H]{\iu \omega_x}} & = - \fx[\arg]{\fx[H]{- \iu \omega_x}}
    \end{align*}
\end{mdframed}

\begin{mdframed}[style=ExampleFrame]
    \begin{example}
    \end{example}
    Una señal periódica $\fx[x]{t}$ con nivel DC nulo y período $T_0$ se puede representar con una suma infinita de exponenciales complejas.
    Sus coeficientes $c_k$ están dados para $k>0$ por:
    \[
        c_k = \frac{4 \iu^k}{k}
    \]

    Obtener el espectro de salida para un sistema con la siguiente función de transferencia:
    \[
        \fx[H]{\iu \omega} =
        \pulse{f \, T_0}{5}
        \, e^\frac{\iu \pi \, f \, T_0}{4}
    \]
    
    \concept{Resolución:}
    
    La sucesión dada es la siguiente:
    \begin{align*}
        \inBraces{c_k}
        & = \inBraces{\cdots \frac{-4}{5 \iu} ; -1 ; \frac{4}{3 \iu} ; 2 ; \frac{-4}{\iu} ; 4 \iu ; -2 ; \frac{-4}{3 \iu} ; 1 ; \frac{4}{5 \iu} \cdots}
        \\
        & = \inBraces{\cdots \frac{4 \iu}{5} ; -1 ; \frac{-4 \iu}{3} ; 2 ; 4 \iu ; 4 \iu ; -2 ; \frac{4 \iu}{3} ; 1 ; \frac{-4 \iu}{5} \cdots}
    \end{align*}

    Pero según el enunciado, la señal $\fx[x]{t}$ está dada por aquellos coeficientes donde $k>0$:
    \[
        \inBraces{c_k} = \inBraces{4 \iu ; -2 ; \frac{4 \iu}{3} ; 1 ; \frac{-4 \iu}{5} \cdots}
    \]

    Notar que el espectro bilateral de $\fx[x]{t}$ está dado por el módulo y el argumento de los coeficientes.
    En el desarrollo en series de $\fx[x]{t}$, la exponencial que tiene la variable temporal es una oscilación simple.
    Son los coeficientes $c_k$ los que determinan el aporte de la magnitud y la fase de cada frecuencia.
    Por ejemplo, para $k=5$ se tiene que el sumando es
    \[
        c_5 \, e^{\iu \omega_0 \, t}
        = \frac{4 \iu}{5} \, e^{\iu \omega_0 \, t}
        = \frac{4}{5} \, e^{\iu \frac{\pi}{2}} \, e^{\iu \omega_0 \, t}
        = \frac{4}{5} \, e^{\iu \inParentheses{\omega_0 \, t + \frac{\pi}{2}}}
    \]
    haciéndose evidente que el módulo y el argumento de $c_5$ son iguales al módulo y la fase de todo el sumando.

    Por lo tanto, el espectro de la señal $\fx[x]{t}$ es
    \begin{center}
        \includegraphics[width=\linewidth]{./images/ej_siemtria_hermitiana_x.png}
    \end{center}

    La función de transferencia es:
    \[
        \fx[H]{\iu \omega} = \pulse{\omega}{5 \, \omega_0} \, e^{-\frac{\iu \, \pi \, \omega}{4 \, \omega_0}}
    \]
    pues
    \[
        \left\{
        \begin{aligned}
            & \frac{f \, T_0}{5}
            = \frac{1}{5} \, f \, T_0
            = \frac{1}{5} \, \frac{\omega}{2 \pi} \, \frac{2 \pi}{\omega_0}
            = \frac{\omega}{5 \, \omega_0}
            \\[1em]
            & \frac{\iu \pi \, f \, T_0}{4}
            = \frac{\iu \pi}{4} \, f \, T_0
            = \frac{\iu \pi}{4} \, \frac{\omega}{2 \pi} \, \frac{2 \pi}{\omega_0}
            = \frac{\iu \, \pi \, \omega}{4 \, \omega_0}
        \end{aligned}
        \right.
    \]

    Luego:
    \begin{gather*}
        \norm{\fx[H]{\omega}} =
        \left\{
        \begin{aligned}
            & 1 \text{ si } \omega \in [-2 \, \omega_0 ; 2 \, \omega_0]
            \\
            & 0 \text{ si } \omega \notin [-2 \, \omega_0 ; 2 \, \omega_0]
        \end{aligned}
        \right.
        \\[1em]
        \fx[\arg]{\fx[H]{\omega}} =
        - \frac{\pi \, \omega}{4 \, \omega_0}
    \end{gather*}

    Y el espectro de la función de transferencia es:
    \begin{center}
        \includegraphics[width=\linewidth]{./images/ej_siemtria_hermitiana_h.png}
    \end{center}

    Multiplicando las magnitudes y sumando las fases de $\fx[x]{t}$ con las de $\fx[H]{\omega}$, se obtiene:
    \begin{center}
        \includegraphics[width=\linewidth]{./images/ej_siemtria_hermitiana_y.png}
    \end{center}
\end{mdframed}

\section{Fase lineal}

\section{Sistemas típicos}

\section{Interconexión de sistemas}

\section{Filtros}