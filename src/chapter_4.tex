\chapter{Diagramas de flujo de señales}

Los diagramas en bloques muestran el flujo de la señal a través de las distintas operaciones que conforman un sistema
Permiten visualizar cómo se transforma la entrada en la salida mediante una secuencia de procesos

A partir de un diagrama de flujo, es posible obtener la respuesta al impulso o la respuesta en frecuencia del sistema completo.
Cada bloque del diagrama puede representarse de distintas formas, equivalentes entre sí:

\begin{itemize}
    \item Por su \textbf{respuesta al impulso} individual $\fx[h_\ith]{t}$
    \item Por su \textbf{respuesta en frecuencia} individual $\fx[H_\ith]{\iu \omega}$
    \item Por la \textbf{operación que realiza} (multiplicación, retardo, derivación, integración, etc.)
\end{itemize}

Un diagrama de flujo compuesto por derivadores, integradores y sumadores puede transformarse en una ecuación diferencial.
A la inversa, un sistema descrito por una ecuación diferencial lineal con coeficientes constantes tiene una representación equivalente en forma de diagrama de flujo.

\section{Respuesta en frecuencia}

Si la entrada al sistema es una exponencial compleja $e^{\iu \omega t}$ se recorre el diagrama reemplazando esta señal y aplicando las operaciones correspondientes.
Como $e^{\iu \omega t}$ es autovector de los sistemas LTI, la salida será de la forma
\[
    \fx[y]{t} = \fx[H]{\iu \omega} \, e^{\iu \omega t}
\]
donde $\fx[H]{\iu \omega}$ es la respuesta en frecuencia del sistema completo.

\section{Composición en el dominio de la frecuencia}

Si cada bloque está representado por su respuesta en frecuencia $\fx[H_\ith]{\iu \omega}$, la respuesta en frecuencia total del sistema, $\fx[\sub{H}{eq}]{\iu \omega}$ , se obtiene:
\begin{itemize}
    \item \textbf{Multiplicando} las respuestas en frecuencia de bloques en \textbf{serie},
    \item \textbf{Sumando} las respuestas de bloques en \textbf{paralelo},
    \item Aplicando las transformaciones correspondientes si hay \textbf{retroalimentaciones}.
\end{itemize}

\section{Respuesta al impulso}

Si la entrada es $\fx[x]{t} = \fx[\delta]{t}$, se recorren los bloques con esta señal y se obtiene la salida $\fx[y]{t}$.
Por definición, $\fx[y]{t} = \fx[h]{t}$ corresponde a la respuesta al impulso del sistema.
Una vez obtenida, cualquier salida ante entrada arbitraria se calcula por convolución:
\[
    \fx[y]{t} = \fx[h]{t} * \fx[x]{t}
\]

\section{Convolución entre bloques}

Es posible que cada bloque del diagrama está representado por su respuesta al impulso $\fx[h_\ith]{t}$.
La respuesta impulsiva total del sistema $\fx[\sub{h}{eq}]{t}$ se obtiene mediante la convolución sucesiva de las respuestas individuales, respetando la estructura del diagrama.
Finalmente, la salida para cualquier entrada $\fx[x]{t}$ se obtiene como
\[
    \fx[y]{t} = \fx[\sub{h}{eq}]{t} * \fx[x]{t}
\]