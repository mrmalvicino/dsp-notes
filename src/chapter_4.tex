\chapter{Caracterización de sistemas}

Los diagramas en bloques muestran el flujo de la señal a través de las distintas operaciones que conforman un sistema.
Permiten visualizar cómo se transforma la entrada en la salida mediante una secuencia de procesos.
A partir de un diagrama de flujo, es posible obtener la respuesta al impulso o la respuesta en frecuencia del sistema completo.

Un diagrama de flujo compuesto por derivadores, integradores y sumadores puede transformarse en una ecuación diferencial.
De manera inversa, un sistema descrito por una ecuación diferencial lineal con coeficientes constantes tiene una representación equivalente en forma de diagrama.

Cada bloque del diagrama puede representarse de distintas formas, equivalentes entre sí:

\begin{enumerate}
    \item Por la \textbf{operación que realiza} en función del tiempo (multiplicación, retardo, derivación, integración, etc.).
    \item Por su \textbf{respuesta en frecuencia} individual $\fx[H_\ith]{\iu \omega}$.
    \item Por su \textbf{respuesta al impulso} individual $\fx[h_\ith]{t}$.
\end{enumerate}

\section{Diagramas de flujo}

Los diagramas de bloques permiten visualizar cómo se procesa una señal dentro de un sistema.
A partir de esta representación, es posible caracterizar el sistema de distintas maneras, según el enfoque que se adopte:

\begin{itemize}
    \item Espectro frecuencial \textbf{por definición}, exitando el sistema con $\fx[x]{t} = \complexExp$.
    \item Respuesta impulsiva \textbf{por definición}, exitando el sistema con $\fx[x]{t} = \fx[\delta]{t}$.
    \item Espectro frecuencial \textbf{por composición} de sistemas individuales $\fx[H_\ith]{\iu \omega}$.
    \item Respuesta impulsiva \textbf{por convolución} de sistemas individuales $\fx[h_\ith]{t}$.
\end{itemize}

Cuando el diagrama muestra explícitamente las \textbf{operaciones} que realiza cada bloque (como derivación, integración o retardo), es posible obtener la respuesta en frecuencia y la respuesta impulsiva del sistema directamente \textbf{por definición}.
Si en cambio el diagrama presenta las \textbf{respuestas en frecuencia individuales} de cada bloque, se puede determinar una respuesta en frecuencia equivalente mediante la \textbf{composición algebraica} de dichas funciones.
Por último, si los bloques están definidos a través de sus \textbf{respuestas impulsivas individuales}, es posible obtener una respuesta impulsiva global mediante \textbf{convoluciones sucesivas}, siempre que el sistema no tenga retroalimentación.

\begin{mdframed}[style=ExampleFrame]
    \begin{example}
    \end{example}
    Dado un sistema LTI descrito por la siguiente ecuación diferencial
    \[
        \deriv[^3]{t^3} \fx[y]{t} + \deriv{t} \fx[y]{t}
        = \deriv[^2]{t^2} \fx[x]{t} - \fx[x]{t}
    \]
    se pretende encontrar:

    \begin{enumerate}
        \item El diagrama de operadores,
        \item La respuesta en frecuencia por definición,
        \item El diagrama de rtas. en frecuencia individuales,
        \item La respuesta en frecuencia por composición.
    \end{enumerate}

    \concept{Diagrama de operadores:}

    Integrando la ecuación diferencial, se obtiene
    \[
        \fx[y]{t} = \deriv{t} \fx[x]{t} - \int_{-\infty}^{t} \fx[x]{\tau} \dif \tau - \deriv[^2]{t^2} \fx[y]{t}
    \]

    De manera que el diagrama de flujo es

    \begin{center}
        \def\svgwidth{\linewidth}
        \input{./images/ej_1_diagramas_de_flujo_operaciones.pdf_tex}
    \end{center}

    \concept{Respuesta en frecuencia por definición:}

    Al evaluar $\fx[x]{t} = \complexExp$ y $\fx[y]{t} = H \, \complexExp$ en la ecuación diferencial, se tiene
    \begin{align*}
        (\iu \omega)^3 \, H \, \complexExp + (\iu \omega) \, H \, \complexExp
        & = (\iu \omega)^2 \, \complexExp - \complexExp
        \\[1em]
        (\iu \omega)^3 \, H + (\iu \omega) \, H
        & = (\iu \omega)^2 - 1
        \\[1em]
        H
        & = \frac{(\iu \omega)^2 - 1}{(\iu \omega)^3 + \iu \omega}
    \end{align*}

    \concept{Diagrama de rtas. en frecuencia individuales:}

    Para obtener el diagrama equivalente, reemplazamos cada operación por las respuestas equivalentes obtenidas en las sección \ref{sec:sistemasTipicos}:

    \begin{center}
        \def\svgwidth{\linewidth}
        \input{./images/ej_1_diagramas_de_flujo_individuales.pdf_tex}
    \end{center}

    \concept{Respuesta en frecuencia por composición:}

    El subsistema de la izquierda tiene un derivador y un integrador en paralelo.
    El subsistema de la derecha, tiene un derivador doble con retroalimentación.
    Estos dos están conectados en cascada.
    Usando las respuestas equivalentes obtenidas en las sección \ref{sec:InterconexionDeSistemas}, se obtiene:

    \begin{align*}
        H
        &= \inParentheses{\iu \omega - \frac{1}{\iu \omega}} \frac{1}{1 + (\iu \omega)^2}
        \\[1ex]
        &= \frac{(\iu \omega)^2 - 1}{\iu \omega} \, \frac{1}{1 + (\iu \omega)^2}
        \\[1ex]
        &= \frac{(\iu \omega)^2 - 1}{\iu \omega + (\iu \omega)^3}
    \end{align*}
\end{mdframed}