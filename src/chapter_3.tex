\chapter{Convolución}
\label{chapter:convolucion}

La convolución modela cómo un sistema acumula los efectos de una señal a lo largo del tiempo.
Pondera cada valor de entrada según la respuesta al impulso del sistema aplicada instante a instante.

Esta operación permite calcular la salida de un sistema LTI ante cualquier señal de entrada.
Para ello, basta con conocer la respuesta al impulso del sistema y realizar la convolución con la señal deseada.

La respuesta al impulso es, por definición, la salida del sistema cuando se lo excita con una delta de Dirac.
Y la salida ante una entrada arbitraria se obtiene convolucionando dicha entrada con la respuesta impulsiva del sistema.

\begin{center}
    \def\svgwidth{0.6\linewidth}
    \input{./images/system.pdf_tex}
\end{center}

\section{Integral de convolución}

Tomando $\fx[\delta]{t-t_0}$ como un funcional para una señal contínua $\fx[x]{t}$, según la propiedad \ref{prop:limDebil}, se tiene
\[
    \int_{-\infty}^\infty \fx[\delta]{t-t_0} \, \fx[x]{t} \, \dif t = \fx[x]{t_0}
\]
y renombrando la variable de integración queda
\[
    \int_{-\infty}^\infty \fx[\delta]{\tau-t_0} \, \fx[x]{\tau} \, \dif \tau = \fx[x]{t_0}
\]

Dado que $\fx[\delta]{t} = \fx[\delta]{-t}$ se tiene para un $t=t_0$ genérico
\[
    \fx[x]{t} = \int_{-\infty}^\infty \fx[\delta]{t-\tau} \, \fx[x]{\tau} \, \dif \tau
\]

Sea $T$ una transformación dada por un sistema LTI.
Al excitar el sistema con $\fx[x]{t}$ se obtiene $\fx[y]{t} = \fx[T]{\fx[x]{t}}$ como salida, dada por
\[
    \fx[y]{t} = \fx[T]{ \int_{-\infty}^\infty \fx[\delta]{t-\tau} \, \fx[x]{\tau} \, \dif \tau }
\]
que por linealidad queda
\[
    \fx[y]{t} = \int_{-\infty}^\infty \fx[T]{\fx[\delta]{t-\tau}} \, \fx[x]{\tau} \, \dif \tau
\]
y por invarianza en el tiempo, $\fx[T]{\fx[\delta]{t-\tau}}$ es la respuesta al impulso (Def. \ref{defn:respuestaAlImpulso}).

\begin{mdframed}[style=DefinitionFrame]
    \begin{defn}
    \end{defn}
    \cusTi{Integral de convolución}
    \[
        \fx[y]{t}
        = \fx[x]{t} * \fx[h]{t}
        = \convol{x}
    \]
\end{mdframed}

\section{Propiedades de la convolución}

\subsection{Existencia del elemento neutro}

La convolución de un impulso con la respuesta impulsiva de un sistema está dada por
\[
    \fx[\delta]{t} * \fx[h]{t} = \convol{\delta}
\]

Pero, según la propiedad \ref{prop:propiedadDeMuestreo}, $\fx[\delta]{\tau}$ pellisca el valor que $h$ toma en $\tau = 0$:
\[
    \fx[\delta]{t} * \fx[h]{t}
    = \int_{-\infty}^\infty \fx[\delta]{\tau} \, \fx[h]{t} \, \dif \tau
\]

Como $\fx[h]{t}$ es independiente de $\tau$, es constante durante la integración:
\[
    \fx[\delta]{t} * \fx[h]{t}
    = \fx[h]{t} \, \int_{-\infty}^\infty \fx[\delta]{\tau} \, \dif \tau
\]
y según la propiedad \ref{prop:integralDeImpulsoUnitario}, la integal vale 1 quedando
\[
    \fx[\delta]{t} * \fx[h]{t} = \fx[h]{t}
\]

Lo cual era de esperarse ya que por definición, si un sistema es alimentado con $\fx[x]{t} = \fx[\delta]{t}$, la salida es la respuesta al impulso.
Y por convolución, la salida tiene que ser la convolución entre $\fx[h]{t}$ y la entrada, que en este caso es $\fx[\delta]{t}$.

Ahora bien, esta operación la podemos hacer para cualesquiera dos funciones.
En el contexto de las señales, interpretamos $\fx[x]{t} = \fx[\delta]{t}$ como la señal de entrada y $\fx[y]{t} = \fx[h]{t}$ como la señal de salida.
Pero si convolucionamos $\fx[\delta]{t}$ con una función genérica $\fx[g]{t}$, llegaríamos a la misma conclusión descrita en la propiedad a continuación.

\begin{mdframed}[style=PropertyFrame]
    \begin{prop}
    \end{prop}
    \[
        \fx[\delta]{t} * \fx[g]{t} = \fx[g]{t}
    \]
\end{mdframed}


\subsection{Desplazamiento en tiempo}

Se tiene la convolución
\[
    \fx[\delta]{t - t_0} * \fx[g]{t}
    = \int_{-\infty}^\infty \fx[\delta]{\tau - \tau_0} \, \fx[g]{t - \tau} \, \dif \tau
\]

Aplicando la propiedad \ref{prop:propiedadDeMuestreo} queda
\begin{align*}
    \fx[\delta]{t - t_0} * \fx[g]{t}
    & = \int_{-\infty}^\infty \fx[\delta]{\tau - \tau_0} \, \fx[g]{t - \tau_0} \, \dif \tau
    \\[1em]
    & = \fx[g]{t - \tau_0} \, \int_{-\infty}^\infty \fx[\delta]{\tau - \tau_0} \, \dif \tau
\end{align*}

y aplicando la propiedad \ref{prop:integralDeImpulsoUnitario} queda
\[
    \fx[\delta]{t - t_0} * \fx[g]{t}
    = \fx[g]{t - \tau_0}
\]

Notar que $\tau_0 = t_0$ fue un cambio de variable hecho para definir la convolución.
Restaurando la notación $t_0$ que originalmente tenía la función $\fx[\delta]{t - t_0}$ se tiene la siguiente propiedad.

\begin{mdframed}[style=PropertyFrame]
    \begin{prop}
    \end{prop}
    \[
        \fx[\delta]{t - t_0} * \fx[g]{t} = \fx[g]{t - t_0}
    \]
\end{mdframed}

\subsection{Propiedad conmutativa}

El orden de las funciones que se convolucionan no altera el resultado.
Podríamos visualizar esta conclusión en los siguientes diagramas, que son equivalentes.

\begin{center}
    \def\svgwidth{0.4\linewidth}
    \input{./images/prop_conmutativa.pdf_tex}
\end{center}

Esta propiedad se puede demostrar haciendo un cambio de variable al aplicar su definición matemática:
\[
    \fx{t} * \fx[g]{t}
    = \convol[g]{f}
    \text{ con } \tau \in \inBrackets{-\infty; \infty}
\]

Sea $\theta = t - \tau$, entonces $\tau = t - \theta$ y $\dif \tau = - \dif \theta$ obteniendo
\[
    \fx{t} * \fx[g]{t}
    = - \int_{\theta = \infty}^{\theta = -\infty} \fx[f]{t - \theta} \, \fx[g]{\theta} \, \dif \theta
\]
que es una expresión equivalente a convolucionar $\fx[g]{t} * \fx{t}$ por definición.

\begin{mdframed}[style=PropertyFrame]
    \begin{prop}
    \end{prop}
    \[
        \fx{t} * \fx[g]{t} = \fx[g]{t} * \fx{t}
    \]
\end{mdframed}

\subsection{Propiedad asociativa}

Esta propiedad permite agrupar las convoluciones sin importar el orden en que se apliquen.
Esto es fundamental en sistemas en serie, como se muestra en el siguiente diagrama.

\begin{center}
    \def\svgwidth{0.8\linewidth}
    \input{./images/prop_asociativa.pdf_tex}
\end{center}

Es posible hacer la convolución entre la entrada y el primer sistema y luego convolucionar el resultado con el segundo sistema.
Y esto sería equivalente a calcular una respuesta al impulso equivalente para ambos sistemas y convolucionarla con la entrada.

Matemáticamente, podemos denotarlo así:
\[
    \inBrackets{\fx[x]{t} * \fx[h_1]{t}} * \fx[h_2]{t}
    = \fx[x]{t} * \inBrackets{\fx[h_1]{t} * \fx[h_2]{t}}
\]

De esta forma, podemos establecer el equivalente a los dos sistemas en serie convolucionando sus respuestas impulsivas.

Observar que en el dominio del tiempo \textbf{no se multiplican} los sistemas en serie como sí pasaba en el dominio de la frecuencia.

\begin{mdframed}[style=PropertyFrame]
    \begin{prop}
    \end{prop}
    \[
        \fx[\sub{h}{eq}]{t} = \fx[h_1]{t} * \fx[h_2]{t}
    \]
\end{mdframed}

\subsection{Propiedad distibutiva}

La respuesta de un sistema LTI ante la suma de dos señales de entrada es igual a la suma de las respuestas individuales de cada señal.
Podemos hacer una analogía con los en sistemas en serie, como se muestra en el siguiente diagrama.

\begin{center}
    \def\svgwidth{0.8\linewidth}
    \input{./images/prop_distributiva.pdf_tex}
\end{center}

Es posible calcular por separado la salida que produce una señal de entrada al atravesar dos sistemas LTI en paralelo, y luego sumar ambas salidas.
O bien, se puede sumar previamente las respuestas al impulso de los dos sistemas, obteniendo así una única respuesta impulsiva equivalente, y luego convolucionarla con la entrada.

Matemáticamente, podemos denotarlo así:
\[
    \fx[x]{t} * \fx[h_1]{t} \pm \fx[x]{t} * \fx[h_2]{t}
    = \fx[x]{t} * \inBrackets{\fx[h_1]{t} \pm \fx[h_2]{t}}
\]

\begin{mdframed}[style=PropertyFrame]
    \begin{prop}
    \end{prop}
    \[
        \fx[\sub{h}{eq}]{t} = \fx[h_1]{t} \pm \fx[h_2]{t}
    \]
\end{mdframed}

\subsection{Derivación en tiempo}

La derivación en tiempo de la convolución está dada por
\[
    \deriv{t} \inBrackets{\fx[f]{t} * \fx[g]{t}} = \deriv{t} \inBrackets{\convol[g]{f}}
\]

Pero $\fx[f]{\tau}$ es independiente del tiempo, por lo que
\[
    \deriv{t} \inBrackets{\fx[f]{t} * \fx[g]{t}}
    = \fx[f]{t} * \deriv{t} \fx[g]{t}
\]
y por propiedad conmutativa, esto es válido para ambas funciones.

\begin{mdframed}[style=PropertyFrame]
    \begin{prop}
    \end{prop}
    \[
        \deriv{t} \inBrackets{\fx[f]{t} * \fx[g]{t}}
        = \deriv{t} \fx[f]{t} * \fx[g]{t}
        = \fx[f]{t} * \deriv{t} \fx[g]{t}
    \]
\end{mdframed}

Esta propiedad es particularmente útil en la práctica a la hora de encontrar la respuesta al impulso de un sistema.
Como no es posible sintetizar una delta de Dirac en la realidad, es posible inyectar un escalón unitario y obtener su derivada:

Si $\fx[x]{t} = \fx[u]{t}$ entonces $\fx[y]{t} = \fx[u]{t} * \fx[h]{t}$, luego
\[
    \deriv{t} \fx[y]{t} = \deriv{t} \fx[u]{t} * \fx[h]{t} = \fx[\delta]{t} * \fx[h]{t} = \fx[h]{t}
\]

\section{Convolución gráfica}

Es posible obtener una expresión para $\fx[y]{t} = \fx[x]{t} * \fx[h]{t}$ analizando las gráficas de $\fx[x]{t}$ y $\fx[h]{t}$ para calcular
\[
    \fx[y]{t} = \convol{x}
\]

Para esto, hay que

\begin{numset}
    \begin{numitem}{Cambiar el dominio de $t$ por $\tau$}
        Generalmente, la integral de convolución está dada para una señal $\fx[x]{t}$ y una respuesta al impulso $\fx[h]{t}$.
        El primer paso consiste en reescribir las funciones en función de $\tau$, ya que la integral se calcula respecto a esta variable:
        \begin{align*}
            \fx[x]{t} & \longrightarrow \fx[x]{\tau}
            \\[1ex]
            \fx[h]{t} & \longrightarrow \fx[h]{\tau}
        \end{align*}
    \end{numitem}

    \begin{numitem}{Invertir y desplazar una de las funciones}
        Luego, seleccionamos una de las funciones para invertirla y desplazarla.
        Se puede elegir tanto $x$ como $h$.
        Es recomendable elegir la más simple: aquella que esté acotada en tiempo o que sea constante.
        Para ser consistente con la notación genérica, se voltea $h$ por motivos didácticos.
        Esto implica hacer el siguiente reemplazo en $h$:
        \[
            \tau \longrightarrow t - \tau
        \]

        Este cambio va a afectar
        \begin{itemize}
            \item eventualmente la fórmula o expresión que tome $h$,
            \item y definitivamente a los intervalos para los que esté definida la función.
        \end{itemize}

        Observar que la función volteada tiene que quedar definida para intervalos de $\tau$, como si $t$ fuese una constante.
    \end{numitem}

    \begin{numitem}{Definir intervalos de $t$}
        ¿Qué es $t$?
        Para la integral, $t$ es una constante.
        Para resolver una convolución gráficamente, en el paso anterior pensamos a $t$ como un corrimiento de $h$ en el tiempo.
        Pero no hay que perder de vista que, $t$, es en realidad la variable independiente que va a tener $\fx[y]{t}$, definida por trozos.
        Es decir que, por motivos prácticos, vamos a calcular distintas integrales como si $t$ efectivamente estuviese variando y $h$ se estuviese moviendo.
        Pero en realidad, $t$ va a ser variable una vez computada la integral.
        Simplemente suponemos lo contrario para saber, \emph{gráficamente}, qué hay que integrar.

        Teniendo esto en cuenta, se definen los intervalos $I_\ith$ de $t$ para los que
        \begin{itemize}
            \item las expresiones que se multiplican en la integral y
            \item los extremos de integración no cambian.
        \end{itemize}

        De esta manera, la integral de convolución queda definida en $\Nth$ tramos:
        \[
            \fx[y]{t} =
            \begin{cases}
                \fx[y_1]{t} \text{si } t \in I_1
                \\
                \fx[y_2]{t} \text{si } t \in I_2
                \\
                \hfil \vdots \hfil
                \\
                \fx[y_\Nth]{t} \text{si } t \in I_\Nth
            \end{cases}
        \]
    \end{numitem}

    \begin{numitem}{Definir extremos de integración}
        Considerar cómo se da el solapamiento en los intervalos definidos.
        Al desplazar $h$, uno o ambos extremos de integración pueden depender de $t$, que es constante para la integral.
        Para encontrar los extremos de integración, pensar qué área estaría calculando la integral si $t$ toma los extremos del intervalo.
    \end{numitem}
\end{numset}

\begin{mdframed}[style=ExampleFrame]
    \begin{example}
    \end{example}
    Dado un sistema cuya respuesta al impulso es $t \, \pulse[1]{t}{2}$, calcular la salida cuando la entrada es $2 \, \fx[u]{t}$.

    \concept{Resolución:}

    Sea
    \[
        \fx[x]{t} = 2 \, \fx[u]{t}
    \]
    la señal de entrada y
    \[
        \fx[h]{t} = t \, \pulse[1]{t}{2}
    \]
    la respuesta al impulso del sistema.

    \begin{numset}
        \begin{numitem}{Cambio de dominio}
            Primero es necesario cambiar el dominio de integración $t \longrightarrow \tau$ quedando
            \begin{align*}
                \fx[x]{\tau}
                & =
                \begin{cases}
                    0 & \text{si } \tau < 0
                    \\
                    2 & \text{si } 0 < \tau
                \end{cases}
                \\[2ex]
                \fx[h]{\tau}
                & =
                \begin{cases}
                    \tau & \text{si } 0 < \tau < 2
                    \\
                    0 & \text{si } \tau \notin [0;2]
                \end{cases}
            \end{align*}
        \end{numitem}

        \begin{numitem}{Inversión y desplazamiento}
            Para voltear y desplazar $h$, hay que reemplazar $\tau$ por $t - \tau$.
            Notar que hay que modificar tanto la expresión de $h$ como el intervalo para el cual se define:
            \[
                0 < \tau < 2
                \longrightarrow
                0 < t - \tau < 2
            \]

            Como la variable de integración es $\tau$, $t$ debe ser tratada como una constante:
            \begin{gather*}
                - t < - \tau < 2 - t
                \\
                t > \tau > t - 2
            \end{gather*}

            Así, luego de voltear y desplazar $\fx[h]{\tau}$, queda
            \[
                \fx[h]{t-\tau}
                =
                \begin{cases}
                    t-\tau & \text{si } t - 2 < \tau < t
                    \\
                    0 & \text{si } \tau \notin [t - 2;t]
                \end{cases}
            \]

            En el siguiente gráfico, podemos visualizar $\fx[x]{\tau}$, $\fx[h]{\tau}$ y $\fx[h]{t-\tau}$ cuando $t=0$.
            \begin{center}
                \def\svgwidth{0.9\linewidth}
                \input{./images/ej_1_convolucion.pdf_tex}
            \end{center}
        \end{numitem}

        \begin{numitem}{Intervalo de $t$}
            Si $t$ es menor a 0, la función $h$ va a estar tan desplazada hacia la izquierda que no va a solaparse con $x$.
            Solo si $t$ toma valores mayores a cero el producto $\fx[x]{\tau} \, \fx[h]{t - \tau}$ va a ser distinto de cero.
            Pero hay que tener en cuenta que el extremo inferior de integración va a cambiar en $t = 2$.
            Por lo tanto, la convolución va a estar definida para los siguientes intervalos:
            \[
                \fx[y]{t} = \fx[x]{t} * \fx[h]{t} =
                \begin{cases}
                    \fx[y_1]{t} & \text{si } -\infty < t < 0
                    \\[1ex]
                    \fx[y_2]{t} & \text{si } 0 < t < 2
                    \\[1ex]
                    \fx[y_3]{t} & \text{si } 2 < t < +\infty
                \end{cases}
            \]
            con $\fx[y_1]{t} = 0$ por no haber solapamiento entre $x$ y $h$.
        \end{numitem}

        \begin{numitem}{Extremos de las integrales}
            Analizando los gráficos para el intervalo $0 < t < 2$.
            Vemos que mientras $t$ vaya de 0 a 2, el extremo inferior de la integral es constantemente 0.
            Mientras que el extremo superior, toma el valor de $t$.
            
            El área calculada para cada valor que tome $t$, va a ser aquella comprendida entre 0 y $t$.
            \[
                \fx[y_2]{t}
                = \int_0^t 2 (t - \tau) \dif \tau
                = t^2
            \]

            Por otra parte, para el intervalo $2 < t < +\infty$, ambos extremos de integración van a estar en función de $t$.
            
            El área calculada para cada valor que tome $t$, va a ser aquella delimitada por el largo del fragmento de recta:
            \[
                \fx[y_3]{t}
                = \int_{t-2}^{t} 2 (t - \tau) \dif \tau
                = 4
            \]
        \end{numitem}
    \end{numset}
\end{mdframed}