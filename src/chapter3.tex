\chapter{Convolución}

\begin{mdframed}[style=MyFrame1]
    \begin{defn}
    \end{defn}
    \cusTi{Respuesta al impulso}
    \cusTe{La respuesta al impulso de un sistema lineal e invariante en el tiempo, representado por una transformación $T$, es la señal de salida obtenida cuando este es excitado por una delta de Dirac como señal de entrada.}
    \begin{equation*}
        \fx[h]{t} = \fx[T]{\fx[\delta]{t}}
    \end{equation*}
\end{mdframed}

Tomando $\fx[\delta]{t-t_0}$ como un funcional para cualquier señal $\fx[x]{t}$ contínua, según la propiedad \ref{prop:limDebil}, se tiene
\begin{equation*}
    \int_{-\infty}^\infty \fx[\delta]{t-t_0} \, \fx[x]{t} \, \dif t = \fx[x]{t_0}
\end{equation*}

Y renombrando la variable de integración queda
\begin{equation*}
    \int_{-\infty}^\infty \fx[\delta]{\tau-t_0} \, \fx[x]{\tau} \, \dif \tau = \fx[x]{t_0}
\end{equation*}

Dado que $\fx[\delta]{t} = \fx[\delta]{-t}$ para cualquier $t=t_0$ genérico
\begin{equation*}
    \fx[x]{t} = \int_{-\infty}^\infty \fx[\delta]{t-\tau} \, \fx[x]{\tau} \, \dif \tau
\end{equation*}

Sea $T$ una transformación dada por un sistema lineal e invariante en el tiempo.

Al excitar el sistema con $\fx[x]{t}$ como señal de entrada se obtiene $\fx[y]{t} = \fx[T]{\fx[x]{t}}$ como salida. Y al aplicar $T$ queda
\begin{equation*}
    \fx[y]{t} = \fx[T]{ \int_{-\infty}^\infty \fx[\delta]{t-\tau} \, \fx[x]{\tau} \, \dif \tau }
\end{equation*}

que por linealidad es
\begin{equation*}
    \fx[y]{t} = \int_{-\infty}^\infty \fx[T]{\fx[\delta]{t-\tau}} \, \fx[x]{\tau} \, \dif \tau
\end{equation*}

y por invarianza en el tiempo $\fx[T]{\fx[\delta]{t-\tau}}$ es la respuesta al impulso, obteniendo la que se conoce como integral de convolución.

\begin{mdframed}[style=MyFrame1]
    \begin{defn}
    \end{defn}
    \cusTi{Integral de convolución}
    \begin{equation*}
        \fx[x]{t} * \fx[h]{t} = \int_{-\infty}^\infty \fx[x]{\tau} \, \fx[h]{t-\tau} \, \dif \tau
    \end{equation*}
\end{mdframed}

Esta operación implica que la salida de un sistema LTI es la \emph{convolución} entre su respuesta al impulso y la señal de entrada.

\begin{mdframed}[style=MyFrame1]
    \begin{prop}
    \end{prop}
    \cusTi{Respuesta al impulso}
    \begin{equation*}
        \fx[y]{t} = \fx[h]{t} * \fx[x]{t}
    \end{equation*}
\end{mdframed}

A continuación se tiene un esquema que representa gráficamente esta interpretación.

\begin{center}
    \def\svgwidth{0.8\linewidth}
    \input{./images/system.pdf_tex}
\end{center}